%% This is the ctufit-thesis example file. It is used to produce theses
%% for submission to Czech Technical University, Faculty of Information Technology.
%%
%% Get the newest version from
%% https://gitlab.fit.cvut.cz/theses-templates/FITthesis-LaTeX
%%
%%
%% Copyright 2021, Eliska Sestakova and Ondrej Guth
%%
%% This work may be distributed and/or modified under the
%% conditions of the LaTeX Project Public Licenese, either version 1.3
%% of this license or (at your option) any later version.
%% The latest version of this license is in
%%  https://www.latex-project.org/lppl.txt
%% and version 1.3 or later is part of all distributions of LaTeX
%% version 2005/12/01 or later.
%%
%% This work has the LPPL maintenance status `maintained'.
%%
%% The current maintainer of this work is Ondrej Guth.
%% Contact ondrej.guth@fit.cvut.cz for bug reports.
%% Alternatively, submit bug reports into the tracker at
%% https://gitlab.fit.cvut.cz/theses-templates/FITthesis-LaTeX/issues
%%
%%

%%%%%%%%%%%%%%%%%%%%%%%%%%%%%%%%%%%%%%%%%
% CLASS OPTIONS
% language: czech/english/slovak
% thesis type: bachelor/master/dissertation
%%%%%%%%%%%%%%%%%%%%%%%%%%%%%%%%%%%%%%%%%
\documentclass[english,master,unicode]{ctufit-thesis}

%%%%%%%%%%%%%%%%%%%%%%%%%%%%%%%%%%
% FILL IN THIS INFORMATION
%%%%%%%%%%%%%%%%%%%%%%%%%%%%%%%%%%
\ctufittitle{Optimization of the painting placement problem using evolutionary techniques} % replace with the title of your thesis
\ctufitauthorfull{Bc.\,Martin Šafránek} % replace with your full name (first name(s) and then family name(s) / surname(s)) including academic degrees
\ctufitauthorsurnames{Šafránek} % replace with your surname(s) / family name(s)
\ctufitauthorgivennames{Martin} % replace with your first name(s) / given name(s)
\ctufitsupervisor{doc.\,RNDr\,Ing\,Marcel Jiřina,\,Ph.D.} % replace with name of your supervisor/advisor (include academic degrees)
\ctufitdepartment{Department of Applied Mathematics} % replace with the department of your defence
\ctufityear{2023} % replace with the year of your defence
\ctufitdeclarationplace{Prague} % replace with the place where you sign the declaration
\ctufitdeclarationdate{\today} % replace with the date of signature of the declaration
\ctufitabstractCZE{
    Tato práce navrhuje genetický algoritmus s novou reprezentací chromozomu a novou implementací křížení
    pro řešení NP těžkého problému umístění obdélníkových obrazů na dvoudimenzionální mřížku.
    Chromozom je reprezentován jako několik stochastických vektorů (vektor obsahující nezáporné prvky, které se sčítají na jedničku).
    Křížení je implementováno jako sčítání vektorů následované normalizací zpět na stochastický vektor.
    Navržené řešení je testováno na vygenerovaném datasetu.
}
\ctufitabstractENG{
    The thesis proposes a genetic algorithm with a novel chromosome representation and novel crossover implementation
    for solving the NP-hard problem of placing rectangular paintings on a two-dimensional grid.
    A chromosome is represented as multiple stochastic vectors (vector that contains non-negative elements that add up to one).
    Crossover is implemented as vector addition followed by normalization back to the stochastic vector.
    The proposed solution is tested on a generated dataset.
}
\ctufitkeywordsCZE{genetické algoritmy, náhodné klíče, struktura chromozomu, řezové stromy, uspořádání pracovišť, FLP, plán polic, optimalizace}
\ctufitkeywordsENG{genetic algorithms, random keys, chromosome structure, slicing trees, facility layout, FLP, shelf-space planning, optimization}
%%%%%%%%%%%%%%%%%%%%%%%%%%%%%%%%%%
% END FILL IN
%%%%%%%%%%%%%%%%%%%%%%%%%%%%%%%%%%

%%%%%%%%%%%%%%%%%%%%%%%%%%%%%%%%%%
% CUSTOMIZATION of this template
% Skip this part or alter it if you know what you are doing.
%%%%%%%%%%%%%%%%%%%%%%%%%%%%%%%%%%

\RequirePackage{iftex}[2020/03/06]
\iftutex % XeLaTeX and LuaLaTeX
\RequirePackage{ellipsis}[2020/05/22] %ellipsis workaround for XeLaTeX
\else
\RequirePackage[utf8]{inputenc}[2018/08/11] %this file encoding
\RequirePackage{lmodern}[2009/10/30] % vector flavor of Computer Modern font
\fi

% hyperlinks
\RequirePackage[pdfpagelayout=TwoPageRight,colorlinks=false,allcolors=decoration,pdfborder={0 0 0.1}]{hyperref}[2020-05-15]

% uncomment the following to hide all hyperlinks
% \RequirePackage[pdfpagelayout=TwoPageRight,hidelinks]{hyperref}[2020-05-15]

\RequirePackage{pdfpages}[2020/01/28]

\setcounter{secnumdepth}{4} % numbering sections; 4: subsubsection


%%%%%%%%%%%%%%%%%%%%%%%%%%%%%%%%%%
% CUSTOMIZATION of this template END
%%%%%%%%%%%%%%%%%%%%%%%%%%%%%%%%%%


%%%%%%%%%%%%%%%%%%%%%%
% DEMO CONTENTS SETTINGS
% You may choose to modify this part.
%%%%%%%%%%%%%%%%%%%%%%
\usepackage{dirtree}
\usepackage{lipsum,tikz}
\usepackage{csquotes}
\usepackage[style=iso-numeric]{biblatex}
\addbibresource{text/bib-database.bib}
\usepackage{listings} % typesetting of sources
% \usepackage{minted} % typesetting of sources

%theorems, definitions, etc.
\theoremstyle{plain}
\newtheorem{theorem}{Věta}
\newtheorem{lemma}[theorem]{Tvrzení}
\newtheorem{corollary}[theorem]{Důsledek}
\newtheorem{proposition}[theorem]{Návrh}
\newtheorem{definition}[theorem]{Definice}
\theoremstyle{definition}
\newtheorem{example}[theorem]{Příklad}
\theoremstyle{remark}
\newtheorem{note}[theorem]{Poznámka}
\newtheorem*{note*}{Poznámka}
\newtheorem{remark}[theorem]{Pozorování}
\newtheorem*{remark*}{Pozorování}
\numberwithin{theorem}{chapter}
%theorems, definitions, etc. END
%%%%%%%%%%%%%%%%%%%%%%
% DEMO CONTENTS SETTINGS END
%%%%%%%%%%%%%%%%%%%%%%

\usepackage{graphicx}
\graphicspath{{figures/}}

% big tables & rotating page
\usepackage{pdflscape}
\usepackage{afterpage}

\usepackage[export]{adjustbox}
%% for aligning: \includegraphics[width=.6\textwidth,left]{example-image}

\usepackage{todonotes}
\usepackage{amsmath}
\usepackage{amsfonts}

% Enable subfloats within figures
\usepackage{subfig}

\usepackage{minted}

% include verbatim in a figure caption
\usepackage{cprotect}

%Add notes under the table, example: https://tex.stackexchange.com/a/13316
\usepackage[flushleft]{threeparttable}

%%%%%%%%%%%%%%%%%%%%%%%%%%%%%%%%%%%%%%%%%%%%%%%%%%%%%%%%%%%%%%%%%%%%%%%%%%%%%%%
% ALGORITHM
%%%%%%%%%%%%%%%%%%%%%%%%%%%%%%%%%%%%%%%%%%%%%%%%%%%%%%%%%%%%%%%%%%%%%%%%%%%%%%%
%\usepackage{algorithm}
%\usepackage{algorithmic}
\usepackage[ruled,vlined]{algorithm2e}
\usepackage{algpseudocode}

%\usepackage{algorithmicx}

%%% nefungujici mess z BP
%\newcounter{nalg}[chapter] % defines algorithm counter for chapter-level
%\renewcommand{\thenalg}{\thechapter .\arabic{nalg}} %defines appearance of the algorithm counter
%\DeclareCaptionLabelFormat{algocaption}{Algorithm \thenalg} % defines a new caption label as Algorithm x.y
%
%\lstnewenvironment{algorithm}[1][] %defines the algorithm listing environment
%{
%    \refstepcounter{nalg} %increments algorithm number
%    \captionsetup{labelformat=algocaption,labelsep=colon} %defines the caption setup for: it ises label format as the declared caption label above and makes label and caption text to be separated by a ':'
%    \lstset{ %this is the stype
%        mathescape=true,
%        frame=tB,
%        numbers=left,
%%		numberstyle=\tiny,
%        basicstyle=\scriptsize,
%        keywordstyle=\color{black}\bfseries\em,
%        keywords={,input, output, return, datatype, function, in, if, else, foreach, while, begin, end, } %add the keywords you want, or load a language as Rubens explains in his comment above.
%        numbers=left,
%        xleftmargin=.04\textwidth,
%        #1 % this is to add specific settings to an usage of this environment (for instnce, the caption and referable label)
%    }
%}
%{}
%%%%%%%%%%%%%%%%%%%%%%%%%%%%%%%%%%%%%%%%%%%%%%%%%%%%%%%%%%%%%%%%%%%%%%%%%%%%%%%

%%%%%%%%%%%%%%%%%%%%%%%%%%%%%%%%%%%%%%
% VLASTNI MAKRA
%%%%%%%%%%%%%%%%%%%%%%%%%%%%%%%%%%%%%%

\def\todoc{\todo[color=yellow]{cite}}
\def\todoi#1{\todo[inline, color=green]{#1}}
\newcommand{\navesti}[1]{\noindent\textbf{\textit{{#1}}}}
\newcommand{\definice}[1]{\textbf{\textit{{#1}}}}
\DeclareMathOperator*{\argmax}{arg\,max}
\DeclareMathOperator*{\argmin}{arg\,min}
\DeclareMathOperator*{\real}{\mathbb{R}}
\DeclareMathOperator*{\realpos}{\mathbb{R}^+}
\DeclareMathOperator*{\nat}{\mathbb{N}}
\DeclareMathOperator*{\natpos}{\mathbb{N}^+}

% minted centering
\usepackage{xpatch,letltxmacro}
\LetLtxMacro{\cminted}{\minted}
\let\endcminted\endminted
\xpretocmd{\cminted}{\RecustomVerbatimEnvironment{Verbatim}{BVerbatim}{}}{}{}

\begin{document}
    \frontmatter\frontmatterinit % do not remove these two commands

    \includepdf[pages={1-}]{assignment-include.pdf} % replace that file with your thesis assignment provided by study office

    \thispagestyle{empty}\cleardoublepage\maketitle % do not remove these three commands

    \imprintpage % do not remove this command

    \tableofcontents % do not remove this command
%%%%%%%%%%%%%%%%%%%%%%
% list of other contents: figures, tables, code listings, algorithms, etc.
% add/remove commands accordingly
%%%%%%%%%%%%%%%%%%%%%%
    \listoffigures % list of figures
    \begingroup
    \let\clearpage\relax
    \listoftables % list of tables
%    \lstlistoflistings % list of source code listings generated by the listings package
    \listofalgorithms
    \listoflistings % list of source code listings generated by the minted package
    \endgroup
%%%%%%%%%%%%%%%%%%%%%%
% list of other contents END
%%%%%%%%%%%%%%%%%%%%%%

%%%%%%%%%%%%%%%%%%%
% ACKNOWLEDGMENT
% FILL IN / MODIFY
% This is a place to thank people for helping you. It is common to thank your supervisor.
%%%%%%%%%%%%%%%%%%%
    \begin{acknowledgmentpage}
        I want to thank my supervisor doc.\,RNDr\,Ing\,Marcel Jiřina,\,Ph.D. for the opportunity to discuss my thesis with him.
        Also, I want to thank my family for their support during my studies and during writing this thesis.
    \end{acknowledgmentpage}
%%%%%%%%%%%%%%%%%%%
% ACKNOWLEDGMENT END
%%%%%%%%%%%%%%%%%%%


%%%%%%%%%%%%%%%%%%%
% DECLARATION
% FILL IN / MODIFY
%%%%%%%%%%%%%%%%%%%
% INSTRUCTIONS
% ENG: choose one of approved texts of the declaration. DO NOT CREATE YOUR OWN. Find the approved texts at https://courses.fit.cvut.cz/SFE/download/index.html#_documents (document Declaration for FT in English)
% CZE/SLO: Vyberte jedno z fakultou schvalenych prohlaseni. NEVKLADEJTE VLASTNI TEXT. Schvalena prohlaseni najdete zde: https://courses.fit.cvut.cz/SZZ/dokumenty/index.html#_dokumenty (prohlášení do ZP)
    \begin{declarationpage}
        I hereby declare that the presented thesis is my own work and that I have cited all
        sources of information in accordance with the Guideline for adhering to ethical
        principles when elaborating an academic final thesis.
        I have used Grammarly to correct minor spelling issues and ChatGPT to assist me with writing Python code that generates graphs. \todo{nechat tuhle jednu vetu zde nebo dat do podekovani?}

        I acknowledge that my thesis is subject to the rights and obligations stipulated by the
        Act No. 121/2000 Coll., the Copyright Act, as amended, in particular that the Czech
        Technical University in Prague has the right to conclude a license agreement on the
        utilization of this thesis as a school work under the provisions of Article 60 (1) of the
        Act.
    \end{declarationpage}
%%%%%%%%%%%%%%%%%%%
% DECLARATION END
%%%%%%%%%%%%%%%%%%%

    \printabstractpage % do not remove this command

%%%%%%%%%%%%%%%%%%%
% SUMMARY
% FILL IN / MODIFY
% OR REMOVE ENTIRELY (upon agreement with your supervisor)
% (appropriate to remove in most theses)
%%%%%%%%%%%%%%%%%%%
%    \begin{summarypage}
%        \section*{Summary section}
%
%        \lipsum[1][1-8]
%
%        \section*{Summary section}
%
%        \lipsum[2][1-6]
%
%        \section*{Summary section}
%
%        \lipsum[3]
%
%        \section*{Summary section}
%
%        \lipsum[2]
%
%        \section*{Summary section}
%
%        \lipsum[1][1-8] Lorem lorem lorem.
%    \end{summarypage}
%%%%%%%%%%%%%%%%%%%
% SUMMARY END
%%%%%%%%%%%%%%%%%%%

%%%%%%%%%%%%%%%%%%%
% ABBREVIATIONS
% FILL IN / MODIFY
% OR REMOVE ENTIRELY
% List the abbreviations in lexicography order.
%%%%%%%%%%%%%%%%%%%


    \chapter{Acronyms}\label{ch:acronyms}

    \begin{tabular}{rl}
        2D-KP  & Two-Dimensional Knapsack             \\
        BL     & Bottom Left                          \\
        BL-F   & Bottom Left Fill                     \\
        BRKGA  & Biased Random Key Genetic Algorithm  \\
        FLP    & Facility Layout Problem              \\
        GA     & Genetic Algorithm                    \\
        I/O    & Input,Output                         \\
        NP     & Nondeterministic Polynomial          \\
        QAP    & Quadratic Assignment Problem         \\
        RKGA   & Random Key Genetic Algorithm         \\
        UA-FLP & Unequal Area Facility Layout Problem \\
    \end{tabular}
%%%%%%%%%%%%%%%%%%%
% ABBREVIATIONS END
%%%%%%%%%%%%%%%%%%%

    \mainmatter\mainmatterinit % do not remove these two commands

%%%%%%%%%%%%%%%%%%%
% THE THESIS
% MODIFY ANYTHING BELOW THIS LINE
%%%%%%%%%%%%%%%%%%%

    \chapter{Introduction}\label{ch:introduction}

Placement of the images or paintings on the wall may seem trivial at first.
However, it is not true.
There are different arrangements and constraints to each particular placement.
For example, an art gallery might want to place paintings on the wall, grouping them by their author, style, etc.
One example of such placement can be seen in figure~\ref{fig:london-wall}.
In addition, the room's lighting and the paintings and wall dimensions need to be considered.
Together, these requirements pose a complex problem to solve.

Furthermore, a solution to the painting placement problem can be used in many other, more practical fields.
For example, a facility layout problem (FLP) places a set of facilities on a grid
while having the same constraints as the painting placement – grouping related facilities and considering their dimensions.
Another field is retail shelf-space planning, which tries to partition a shelf into rectangles.
Subsequently, the partitioned shelf is filled with goods that the customers can buy.
Similarly to the lightning conditions for paintings, goods depend on the
particular placement on the shelf – goods close to the customer's eye level increase their visibility and thus lead to more sales.
Last but not least, there is a plethora of packing problems.
One such well-known problem is the 2D rectangular packing problem, which can be reformulated as a knapsack problem.

This thesis proposes a genetic solution to the painting placement problem based on a slicing tree.
It takes a different approach to input parameters and constraints as opposed to some of the
problems mentioned above – inputs are rectangles (paintings) with fixed width and height,
while the same is true for the wall they are placed on. Also, paintings cannot be rotated.
Lastly, there is no constraint on the solution, i.e., the proposed genetic solution can
produce a placement with overlapping paintings or paintings with their parts outside the wall.
A precise definition of the painting placement problem and other mentioned methods can be found in later chapters.

\todo{strucny popis vsech kapitol a struktury prace} Etiam sapien elit, consequat eget, tristique non, venenatis quis, ante. Nemo enim ipsam voluptatem quia voluptas sit aspernatur aut odit aut fugit, sed quia consequuntur magni dolores eos qui ratione voluptatem sequi nesciunt. Suspendisse nisl. Mauris metus. Cras elementum. Etiam ligula pede, sagittis quis, interdum ultricies, scelerisque eu. Fusce suscipit libero eget elit. Cras elementum. Quisque tincidunt scelerisque libero. Maecenas ipsum velit, consectetuer eu lobortis ut, dictum at dui. Cum sociis natoque penatibus et magnis dis parturient montes, nascetur ridiculus mus. Ut tempus purus at lorem. Nulla est. Etiam posuere lacus quis dolor. Maecenas sollicitudin. Mauris tincidunt sem sed arcu. Duis viverra diam non justo. Pellentesque arcu.

\begin{figure}
    \includegraphics[width=1\textwidth, left]{london_gallery_wall}
    \caption[Painting placement at the The London National Gallery]{Painting placement at the The London National Gallery. Source~\cite{ScreenshotWallGoogle}:}
    \label{fig:london-wall}
\end{figure}





\chapter{Literature review}\label{ch:literature-review}
\chapter{Problem statement and formulation}\label{ch:problem-statement-and-formulation}

This chapter defines the painting placement problem, and its constraints.
First, let us define what a painting placement instance is.

\navesti{Painting placement isntance} is an ordered quadruple

\[
    I = (P, F, L, \pi)\,,
\]

where $P \subseteq (\nat^+, \nat^+)$ are painting dimensions (width$\times$height pairs),
$F \subseteq \real^{|P|, |P|}$ is matrix defining flow between paintings,
$L \in \nat^+ \times \nat^+$ is layout dimension (width$\times$height pair)
and $\pi: \real \times \real \to \real$ is evaluation function.

An example of a painting placement instance is

\[
    I_1 = (\overbrace{((5,4),(8,5))}^{paintings},
    \overbrace{\begin{pmatrix}
                   0   & 5.8 \\
                   5.8 & 0
    \end{pmatrix}}^{flow},
    \overbrace{(15,7)}^{layout},
    \overbrace{f(x,y) = x+y)}^{evaluation function}\,.
\]

Instance $I_1$ has two paintings, the first one with dimensions $5\times4$, the second with $8\times5$.
The flow between them is $5.8$. The layout to which paintings are placed has a width $15$ and a height $7$.
The evaluation function is $f(x,y) = x+y$.

Next, each painting placement instance can have multiple solutions.

\navesti{Painting placement solution} is an ordered set of placement points
\[
    S \subseteq (\natpos \times \natpos)^N  \,,
\]
where $N$ is the size of the painting placement instance, i.e., the number of paintings.
For example, one solution $S_1$ for the instance $I_1$ can be

\[
    S_1 = ((0,0), (6,1)) \,.
\]

This means that the first painting's lower left corner will have coordinates $(0,0)$
and similarly $(6,1)$ for the second painting.
Illustration of both $I_1$ and $S_1$ can be seen in figure~\ref{fig:painting-placement-solution}.

\begin{figure}
    \includegraphics[width=0.9\textwidth, left]{painting_placement_solution}
    \caption[Example of a painting placement solution.]{Example of a painting placement solution $S_1 = ((0,0), (6,1))$ for instance $I_1$.}
    \label{fig:painting-placement-solution}
\end{figure}

\newpage

Lastly, the painting placement solution needs to be evaluated.
Let's assume an instance $I$ and a set of all possible solutions $S$ to that instance.
Then, the painting placement problem is to find a minimum of a cost function $c: S \to \realpos$,
which can also be called an objective function, as


%Let's define $\Omega$ as a set of all possible placement layouts for a given painting placement instance
%and $\omega$ as all possible placement points for paintings in this instance.
%Thus, if the instance size is $N$, i.e., there are $N$ paintings,
%$\Omega = $
%
%For example, one placement layout from $\Omega$ can be seen on figure~\ref{TODO}
%
%and $\omega$ contains all positions where a painting can be placed.
%Optimization problem can be thus defined using $c$ as

\begin{equation}
    \argmin_{x \in S} c(x) = \sum\limits_{i=1}^N\sum\limits_{j=i+1}^N f_{i,j}d(i, j) + \sum\limits_{i=1}^N \pi(i) + \lambda m(x) + \gamma n(x) \,,
\end{equation}

where $f_{i,j}$ is flow between painting $i$ and $j$, $d(i,j)$ is their distance,
$\pi(i)$ is evaluation function applied to painting's $i$ lower left corner,
$m$ calculates number of overlapping paintings with penalization parameter $\lambda \in \real^+$
and $\gamma$ calculates the number of paintings placed outside of their allocated are
with penalization parameter $\gamma \in \real^+$.
\chapter{Coding and layout construction}\label{ch:coding-and-layout-construction}

This chapter is the central part of the thesis as it describes the genetic approach to the painting placement problem.

\section{Genetics}\label{sec:genetics}

This section describes important genetic terms that are used throughout the thesis.
They are allele, gene, chromosome, individual, population, crossover, mutation, and reproductive plan.
Also, in subsection~\ref{subsec:schema-theorem}, the integral part of genetics called the Schema Theorem is described.

First, it is essential to describe what the genetic approach means.
The genetic approach was first introduced by Holland in 1975
to solve optimization problems where it is computationally infeasible to
find an optimal solution by enumerating all possible solutions~\cite{hollandAdaptationNaturalArtificial1975}.

This genetic approach is inspired by nature and Darwin’s Theory of Evolution
– a population of individuals evolving over time.
Individuals more adapted to the environment are
more likely to survive and thus pass their genes to the next generation.
Thus, over time, the population should converge to the state where the adaptation to the environment is the highest~\cite{darwinOriginSpeciesMeans2009}.

Holland in~\cite{hollandAdaptationNaturalArtificial1975} defines several structures that reassemble this natural process.
The most important ones are described in the rest of this section.\\

\navesti{Allele} represents a concrete value that a gene can have.
It can be thus described as a set of alternatives to choose from. \\

\navesti{Gene} is a structure composed of alleles.
It often describes one trait or characteristic. \\

\navesti{Chromosome} is a structure composed of genes.
Thus it is an amalgam of characteristics described by genes.\\

\navesti{Individual} is defined by its chromosome and represents a solution to the problem or a structure from which a solution can be constructed.
A numeric value called \definice{fitness} can be assigned to each individual, representing how well the individual performs in an environment. \\

\navesti{Population} is a set of individuals.
It can be interpreted as a subset of possible solutions.\\

One concrete example of the above-mentioned definitions is in figure~\ref{fig:population}.
There are two individuals in the population, with their chromosomes having two genes.
The first gene is a vector containing permutation with alleles of 1 to 5.
The second gene is a string vector
with alleles having values $H$ or $V$ (cut types from subsec.~\ref{subsec:individual-decoding}).
Lastly, each individual has a fitness value.
Thus, because A’s fitness $30.5$ is greater than B’s fitness $9.8$, we can say that individual A performs better than B.


\begin{figure}
    \includegraphics[width=1.1\textwidth, left]{population}
    \caption[Population example]{Example of a population composed of two individuals.}
    \label{fig:population}
\end{figure}

For the structures defined above, multiple operations called \definice{genetic operators} or simply operators
are defined by Holland and used by other researchers following his work.
Genetic operators aim to create new individual/s using individuals already present in a population as input.
Two of them that are used in this thesis are described below.\\


\navesti{Crossover} genetic operator takes two individuals as input and, by recombination
of their alleles in their genes, produces a new individual/s called \definice{offspring} or \definice{child}.\\

\navesti{Mutation} genetic operator takes one individual as input and produces a new one
which may have some of its alleles replaced by different ones at random.\\

Additionally, there needs to be a process that transforms a population
to a new one.
This process is called the reproductive plan.
Also, there is a special term for the population to which the reproductive plan is applied.
\\

\navesti{Reproductive plan} is a process that takes a population on input and produces a modified population on the output
by using mainly genetic operators.\\

\navesti{Initial population} is the population before the first application of the reproductive plan.\\

\navesti{Generation} $k$ is the population after applying the reproductive plan $k$-times to it.\\

\newpage

Finally, a genetic algorithm uses all the processes mentioned above to
find a solution to some problem.\\

\navesti{Genetic algorithm} or \definice{genetic approach} applies a reproductive plan
on an initial population until the stopping condition is met with the
goal of finding a (sub)-optimal solution to the problem.\\

One example of a genetic approach is in figure~\ref{fig:reproductive-plan}.
At the start, an initial population of individuals is generated.
Then, the reproductive plan is applied until the stopping condition is met.
Application of the reproductive plan creates the next generation by using crossover and mutation genetic operators.

\begin{figure}[h]
    \includegraphics[width=0.65\textwidth, center]{reproductive_plan}
    \caption[Example of a genetic approach]{
        Example of a genetic approach.
        It uses a reproductive plan that creates the next generation by applying crossover and mutation.}
    \label{fig:reproductive-plan}
\end{figure}

\subsection{Schema Theorem}\label{subsec:schema-theorem}

Holland in~\cite{hollandAdaptationNaturalArtificial1975} proposed
the Schema Theorem arguing why the genetic approach described above works.
This subsection describes the main idea behind the argument.
First, an important term to describe is schema.\\

\navesti{Schema} is an extended representation of chromosome,
where each gene can contain a “don’t care” symbol marked as underscore $\_$.
This symbol can take up any value that an allele can in the given context.
We can then say that a chromosome belongs to a schema
and that a schema contains a chromosome.\\

Schema can be illustrated on a chromosome with one gene represented as a vector that contains a permutation of numbers $1$ to $7$.
Then, example of a schema is $H_1 = \langle 5, \_, \_, 2, \_, 3, \_ \rangle$.
It contains $24$ chromosomes, with one example being $\langle 5, 4, 1, 2, 6, 3, 7 \rangle$.
On the other hand, schema $H_2 = \langle 1, 2, 3, 4, 5, 6, \_ \rangle$ contains only one chromosome.

\newpage
There are two other properties that a schema has.
They are length and order and are defined as follows.\\

\definice{Length} of a schema is the distance from the first “non-don’t care” symbol to the last.\\

\definice{Order} of a schema is the number of “non-don’t care” symbols contained in the schema.\\

For example, $H_1$ has length $6$ and order $3$.
It is graphically illustrated in figure~\ref{fig:schema}.
On the other hand, schema $H_2$ has the same length, $6$, but higher order, which is also $6$.

With schema being defined, we can interpret any population of individuals as a pool of schemata.
It can then be reformulated that a genetic approach, which has a reproductive plan and
genetic operators, (a) creates new schemata by recombination of the one already present in the population,
(b) creates schemata that are absent in the population, and (c) keeps a history of the best schemata found.
The Schema Theorem can then be written as

\begin{equation}
    \mathrm{E}[M(H, t+1)] \geq M(H, t) \dfrac{\mu(H)}{\overline{\mu}}\left[ 1 - p_c \dfrac{\delta(H)}{l-1} - \sigma(H)
    p_m \right]\,,
    \label{eq:schema-theorem}
\end{equation}

where $M(H, t)$ is expected number of individuals whose chromosome belongs to schema $H$ in population $t$,
$\mu(H)$ is average fitness of individuals whose chromosome belongs to $H$,
$\overline{\mu}$ is average population fitness,
$\delta(H)$ is length of schema $H$ with it’s maximum length $l$,
$\sigma(H)$ is order of $H$,
$p_c$ is crossover probability, and
$p_m \ll 1$ is mutation probability.

Inequality~\ref{eq:schema-theorem} says, that the success of a schema $H$,
considering only crossover and low probability mutation are purely determined
by its better-than-average performance, length, and order.
It can be thus said that the genetic approach favors
short schemata with low order that have better-than-average performance.

The reasoning behind the argument is that schemata with high order are more likely
to be damaged by mutation, i.e., an allele of a schema is replaced by a different one.
Also, longer schemata are more likely to be split using a crossover, whereas Holland
considers a one-point-crossover that produces an offspring’s chromosome by copying
of the first parent’s chromosome up to the crossover point, followed by the second parent chromosome after the crossover point.

\begin{figure}[h]
    \includegraphics[width=0.7\textwidth, center]{schema}
    \caption[Example of a schema]{Example of a schema, where “don’t care” symbol marked as underscore $\_$.}
    \label{fig:schema}
\end{figure}

\section{Coding}\label{sec:coding}

The central part of a genetic approach to the painting placement problem is how an individual is represented.
This is important not only for the construction of the genetic operators, e.g., crossover and mutation but also for the process of decoding an individual from its representation to the solution.
In this thesis, a novel individual representation is introduced.

An individual is represented as a 3D chromosome—which means having three genes—that is composed of
(1) painting sequence random key, (2) slicing order random key, and (3) orientation probabilities.
An example of a chromosome is in figure~\ref{fig:chromosome}

Let us use the notation for painting sequence random key as $PS_{rk}$,
slicing order random key as $SO_{rk}$,
orientation probabilities as $OR_{prob}$ and instance size as $N$, i.e., number of paintings.
First two are vectors, where $PS_{rk} \in \real^N$ and $SO_{rk} \in \real^{N-1}$.
Orientation probabilities is a matrix where $OR_{prob} \in \real^{N-1, N-1}$.
Multiple constraints apply to each of these parts with
a stochastic vector meaning a vector that contains non-negative elements that add up to one.

\begin{itemize}
    \item Painting sequence random key is a stochastic vector.
    \item Slicing order random key is a stochastic vector.
    \item Each row in orientation probabilities is a stochastic vector.
\end{itemize}


The above-mentioned representation is based on a solution to the facility layout problem using a slicing tree representation
from~\cite{friedrichIntegratedSlicingTree2018, riponAdaptiveVariableNeighborhood2013},
where the authors also represent an individual as a 3D chromosome.
However, the difference is that in this thesis, each part is a stochastic vector while
in~\cite{friedrichIntegratedSlicingTree2018, riponAdaptiveVariableNeighborhood2013} authors use concrete identifiers of facilities, slicing order, and orientation.
Thus, an additional decoding layer is introduced together with different genetic operators.

\begin{figure}[htp]
    \includegraphics[width=0.8\textwidth, center]{chromosome}\caption[Example of an individual representation]{
        Example of an individual representation – two vectors and one matrix.
        Each vector and matrix row form a stochastic vector, i.e., they contain non-negative elements that add up to 1.
    }
    \label{fig:chromosome}
\end{figure}

There are multiple ideas behind representing an individual as a set of stochastic vectors that stem
from extending RKGA~\cite{beanGeneticAlgorithmsRandom1994}, where chromosome is represented as a vector of values from $[0,1]$. \todo{nahradit zavorky [ a ] za jine, je to matouci s citacemi}
First of them is the ability to perform mutation at an arbitrary element of these vectors using
a simple replacement, i.e., substituting an element for a random one from interval $[0,1]$,
followed by normalization back to the stochastic vector.
For example, using representation described in~\cite{friedrichIntegratedSlicingTree2018, riponAdaptiveVariableNeighborhood2013},
there has to be a different mutation method for each part of a chromosome.
Using the representation proposed in this thesis, there has to be only one mutation operator that can be used universally for all parts of the chromosome.

Additionally, when using representations similar to~\cite{friedrichIntegratedSlicingTree2018, riponAdaptiveVariableNeighborhood2013},
after the application of the genetic operators, usually crossover and mutation, an invalid individual might be created.
That is an individual that does not represent any solution.
The presence of invalid individuals might lead to performance loss in facility layout problem~\cite{liuMultiimprovedGeneticAlgorithm2012}
Moreover, unique solutions for dealing with invalid individuals must be introduced, for example, left-to-right scan used by~\cite{hwangGeneticAlgorithmApproach2009, kandasamyEffectiveLocationMicro2020}.
The solution proposed in this thesis produces only valid individuals.

Finally, the reasoning behind using a stochastic vector and not $[0,1]^N$ as in RKGA~\cite{beanGeneticAlgorithmsRandom1994}
is the unique implementation of crossover used in this thesis, which is described in~\cite{subsec:crossover}.


\newpage


\section{Layout construction}\label{sec:layout-construction}

This section describes the process of how to transform or decode an individual.
This transformation aims to create a solution to the painting placement problem,
which is a set of placement points for the paintings.
There are multiple steps to this process.

\begin{enumerate}
    \item Individual decoding (\ref{subsec:individual-decoding}).
    \item Slicing tree construction (\ref{subsec:slicing-tree-construction}).
    \item Slicing layout construction (\ref{subsec:slicing-tree-construction}).
    \item Using placement heuristic to create a painting placement solution (\ref{subsec:placement-heuristics}).
\end{enumerate}

Steps in the transformation of an individual to the painting placement solution are
in figure~\ref{fig:layout-construction-steps}.
All of them steps are explained in the following text.

\begin{figure}[h!]
    \includegraphics[height=0.65\textheight, center]{layout_construction_steps}
    \caption{Steps in the transformation of an individual to the painting placement solution.}
    \label{fig:layout-construction-steps}
\end{figure}

\subsection{Individual decoding}\label{subsec:individual-decoding}
First, we must decode an individual to the representation from which a slicing layout can be constructed.
A decoded individual is composed of (1) painting sequence, (2) slicing order, and (3) orientations.
An example of individual decoding is in figure~\ref{fig:individual-decoding}.

Let us use the notation for painting sequence as $PS$, slicing order as $SO$, orientations $OR$.
$PS$ contains painting identifiers, $SO$ contains information used to construct slicing layout,
and $OR$ contains type of the cuts in the slicing layout – $H$ for horizontal, $V$ for vertical
and $*$ for wildcard, that can take up any value $H$ or $V$.

\subsubsection*{Random keys decoding}
Decoding both $PS_{rk}$ and $SO_{rk}$ is the same as the RKGA in~\cite{beanGeneticAlgorithmsRandom1994}.
The graphical illustration is in figure~\ref{fig:individual-decoding} marked as~\textit{random key decoder}.
Decoding random keys can be explained in the following steps on a sequence of four numbers $S = 0.3, 0.2, 0.4, 0.1$\,.

\begin{enumerate}
    \item Create $S'$ by adding a lower index to each element from $S$, which marks its ordinal position starting from one.
    $S' = 0.3_1, 0.2_2, 0.4_3, 0.1_4$\,.
    \item Sort $S'$ in descending order. $S' = 0.1_4, 0.2_2,  0.3_1, 0.4_3$\,.
    \item Take lower indexes of $S'$.
    It is the result – $4, 2, 1, 3$.
\end{enumerate}

\subsubsection*{Orientation probabilities decoding}

Last part of the individual, matrix $OR_{prob} \in \real^{N-1, 3}$, decodes to the sequence of symbols from $\Sigma_* = \{H, V, *\} $.
The graphical illustration is in figure~\ref{fig:individual-decoding} marked as~\textit{orientation decoder}.

Decoding $OR_{prob}$ translates each row to one symbol from $\Sigma_*$, producing a symbol sequence.
Thus, decoding orientation probabilities can be explained for one row, say $R = 0.7, 0.2, 0.1$
in the following steps.


\begin{enumerate}
    \item Create $R'$ by adding lower index $H$ to the first, $V$ to the second, and $*$ to the last $R$'s elements.
    $R' = 0.7_H, 0.2_V, 0.1_*$
    \item Select element from $R'$ with the maximum value. $\max R' = 0.7_H$.
    \item Take lower index of $\max R'$.
    It is the result – $H$.
\end{enumerate}

There is one exception to the steps described above.
It is the limit on the maximum number of $*$ symbols produced by orientation probabilities decoding.
Let us call this limit $k$.
If the limit is not applicable, i.e., $k \geq N-1$, there is no change to steps 1--3 described above.
However, if applicable, only the first $k$ symbols $*$ with the highest value are considered.
It is achieved by setting value to $0$ (only for the duration of the decoding) to the bottom $N-1-k$ symbols $*$ with the lowest values.
Then the same 1--3 steps are applied as described above.

One example where the limit applies is for the $k=1$ and $OR_{prob}$ that has two rows, $R_1 = 0.2, 0.3, 0.5$ and $R_2 = 0.1, 0.2, 0.7$.
Without exception, the result is $*, *$.
However, considering the exception on the maximum limit $k=1$, the result is $V, *$.
Reason is that in $R_2$, symbol $*$ has value $0.7$,
which is higher than value of $*$ in $R_1$, which is $0.5$.

\begin{figure}[h!]
    \includegraphics[width=1.1\textwidth, left]{individual_decoding}
    \caption{
        Individual decoding example. Both the painting sequence random key and slicing order random key
        are decoded using the same procedure. The decoded individual can be used to construct a slicing tree.
    }
    \label{fig:individual-decoding}
\end{figure}

\subsection{Slicing layout construction}\label{subsec:slicing-tree-construction}
In the previous subsection, decoding an individual is described.
The decoded individual consists of three parts – painting sequence, slicing order, and orientations.
With this representation, a slicing layout can be constructed.
This construction has three steps, which are as follows.

\begin{enumerate}
    \item Construct unresolved slicing tree from a decoded individual.
    \item Resolve an unresolved slicing tree.
    \item Create a slicing layout using a resolved slicing tree.
\end{enumerate}

\subsubsection*{Slicing tree construction}

First, let us describe what a slicing tree is.
The slicing tree was first introduced in 1982 by Otten~\cite{ottenAutomaticFloorplanDesign1982} to solve automatic floorplan design.
In the most general sense, it is a tree that recursively divides space into rectangles.
This thesis defines and uses the slicing tree in two variants.\\

\navesti{Resolved slicing tree} is a tree with internal nodes having values from $\Sigma = \{H, V\}$
and leafs having values from the painting sequence.\\

\navesti{Unresolved slicing tree} is an extension of a resolved slicing tree where internal nodes have values from $\Sigma_* = \{H, V, *\}$ \\

Symbols $H$ and $V$ are common for both types of the slicing tree.
$H$ means a horizontal cut and $V$ means a vertical cut.
An unresolved slicing tree can also contain the wildcard symbol $*$.
This symbol means a cut that best suits the use – wildcard can thus become both $H$ or $V$.
The presence of a wildcard symbol is a novel idea also proposed in this thesis. \todo{vice zduraznit}

Now, we can use a decoded individual to construct an unresolved slicing tree.
This construction is graphically illustrated in the left part of a figure~\ref{fig:slicing-tree-construction}.
During this process, the painting sequence results in leaf nodes, slicing order determines the shape of a tree, and orientations are the values assigned to internal nodes.
Thus, each decoded individual represents one unresolved slicing tree.

Finally, an unresolved slicing tree is resolved.
Resolving is graphically illustrated in the right part of a figure~\ref{fig:slicing-tree-construction},
where the unresolved tree contains one wildcard symbol $*$ as a root.
By resolving this tree, $*$ is first replaced by $H$ and then by $V$.
In this case, resolving the unresolved slicing tree produces two resolved slicing trees,
which differ in root node value.
In the general case, an unresolved slicing tree can at most resolve to $2^k$ resolved slicing trees,
where $k$ is the number of internal nodes, i.e., the nodes that can contain wildcard $*$.
Reformulation for a decoded individual is that decoded individual can, at most, represent
$2^k$ resolved slicing trees, where $k$ is the number of orientations.


\afterpage{%
    \clearpage% Flush earlier floats (otherwise order might not be correct)
    \begin{landscape}% Landscape page
        \begin{figure}[]
            \centering
            \includegraphics[width=1.5\textwidth]{slicing_tree_construction}
            \caption{
                On the left is an example of unresolved slicing tree construction from a decoded individual.
                On the right is an example of resolving an unresolved slicing tree.}
            \label{fig:slicing-tree-construction}
        \end{figure}
    \end{landscape}
    \clearpage% Flush page
}

\subsubsection*{Slicing layout}

Next, we can construct a slicing layout using a resolved slicing tree.\\

\navesti{Slicing layout} is the recursive partitioning of space to rectangles.\\

Creating a slicing layout from a slicing tree can be explained using an
example in figure~\ref{fig:slicing-layout-dimensions}.
We are given a space to partition. Let us call it layout. Also, we are given areas of rectangles to place.
Rectangles to place are the paintings, and the layout is the wall on which the paintings are being placed.
The example has rectangles $1,2,3$ with areas $a_1, a_2, a_3$.
This is depicted in the left part of the figure.
Then, we recursively traverse the resolved slicing tree.
Depending on the node value, there are three possible actions.

\begin{itemize}
    \item $H$ – cut layout horizontally.
    \item $V$ – cut layout vertically.
    \item Otherwise, assign node value to the layout.
\end{itemize}

After performing the cut, the process mentioned above is recursively repeated
for the left and right child.
If the cut is horizontal, the left child is given the upper part of the cut as its layout, and the right child is given the lower part.
If the cut is vertical, the left child is given the left part of the cut as its layout, and the right child is given the right part.
This can be seen in the middle part of the figure, where the cut is vertical.
The left child is an orphan, i.e., it has no children.
Thus, the left part of the cut is assigned value 1.
The right child is not an orphan, meaning the process is applied recursively to the right part of the cut and the right child.
It is depicted on the right part of the figure.

The last part of creating a slicing layout is the position of the cut points.
As mentioned above, there are two cut types – horizontal and vertical.
For each cut type, the position is determined proportionally to the area of rectangles assigned to the cut result.
Again, it can be described using an example in figure~\ref{fig:slicing-layout-dimensions}.
The first cut is vertical, where the left part of the cut is assigned rectangles $1$ and the right part is assigned rectangles $2,3$.
The vertical cut thus splits the layout into two parts – the left part having $1/3$ of the total layout area and the right part
having the rest.


\afterpage{%
    \clearpage% Flush earlier floats (otherwise order might not be correct)
    \begin{landscape}% Landscape page
        \begin{figure}
            \centering
            \includegraphics[width=1.5\textwidth]{slicing_layout_dimensions}
            \caption{Example of a slicing layout construction from a resolved slicing tree. There are three paintings, $1,2,3$ together with their areas $a_1, a_2, a_3$.
            The position of a cut is determined proportionally to the area of the paintings.} \label{fig:slicing-layout-dimensions}
        \end{figure}
    \end{landscape}
    \clearpage% Flush page
}


\newpage

\subsection{Placement heuristics}\label{subsec:placement-heuristics}



\chapter{Computational results}\label{ch:computational-results}

This chapter presents the computational results of the proposed solution,
generated dataset, hyperparameters of the genetic algorithm,
and implementation of the computation server
to which a user can submit a painting placement instance and receive a solution to that instance.

First, four testing scenarios used for evaluation are described in
section~\ref{sec:scenarios}.
They are random, clustering, packing, and London National Gallery.
Next, in section~\ref{sec:dataset}, the dataset created for each testing scenario is described.
Then, section~\ref{sec:hyper-parameters} describes and discusses
the hyperparameters of the proposed genetic algorithm~\ref{alg:genetic}.
Also, reasonable hyperparameter values are determined.
Section~\ref{sec:results} presents a painting placement solutions to the painting placement instances and their visualizations.
Lastly, section~\ref{sec:implementation} describes the implementation of the computation server.

\section{Scenarios}\label{sec:scenarios}

Four testing scenarios evaluate different aspects of the proposed solution.
They are random, clustering, and packing.
Additionally, one scenario describes the painting placement at the London National Gallery in figure~\ref{fig:london-wall}. \\

\navesti{Random scenario} contains randomly generated painting placement instances.
It is mainly used for performance testing.
\\

\navesti{Clustering scenario} tests the ability to form clusters.
It is achieved by dividing the paintings into groups.
Paintings belonging to the same group have increased flow between them.
Paintings from the distinct group have flow between them set to 0.
\\

\navesti{Packing scenario} is the same as the random scenario, with the only difference being that the layout area is equal to the area of all paintings summed together.
It tests the ability to create compact solutions.
\\

\navesti{London National Gallery scenario} contains one painting placement instance created from the London National Gallery in figure~\ref{fig:london-wall}.
It tests the ability to work with actual painting placement used at a gallery.
\\

\section{Dataset}\label{sec:dataset}

As mentioned in Chapter~\ref{ch:literature-review}, there are no datasets in the
literature that would satisfy the definition ~\ref{eq:painting-placement-instance} of painting placement instance.
Thus, all datasets are exclusively created by the author and can be used by other researchers
for benchmarking their solutions.
Generation is performed using a Python programming language in combination with Jupyter Notebook.
Both the datasets and Python code can be found in the attached medium.

\subsection{Generation parameters}\label{subsec:generation-parameters}

Table~\ref{tab:scenarious-params} describes the generation parameters of each testing scenario.
Value \textit{layout area ratio}
describes the ratio between the area of the layout and the painting area sum.
It can thus be computed as

\[
    \dfrac{\sum\limits_{i=1}^{N} w_i h_i}{WH}\,,
\]

where $w_i$ is width, $h_i$ is height of painting $i$, $W$ is width, and $H$ is height of the layout.
If the \textit{layout area ratio} is set to $1$, it means a preference for more compact layouts. On the other hand,
increasing this value implies the presence of free space in the resulting layout.

The value \textit{max painting ratio} controls the maximum aspect ratio between width $w$ and height $h$ of each painting.
It is computed as

\[
    \dfrac{\max(w,h)}{\min(w,h)}\,.
\]

Increasing \textit{max painting ratio} implies the possibility for the generation of paintings
that are very thin, i.e., $w \ll h$ or $h \ll w$. On the other hand, setting the value to 1
implies that every generated painting is square.

The last parameter, \textit{eval function}, determines the evaluation of the space inside the layout.
For example, in the {random scenario, the function is set to $f(x,y) = x+y$ because of its simplicity, linearity, and interpretability.
This means that it is advantageous to place small paintings close to the top right corner as the function value is highest there and
big paintings to the bottom left corner.
On the other hand, for clustering and {packing scenario, it is set to a constant value $f(x,y) = 0$.
The reason is that in those scenarios, different
capabilities are tested (clustering and packing), and using a non-constant function might make
it challenging to interpret the results.

Rest of the parameters, \textit{max painting width}, \textit{max painting height}, \textit{flow min}, \textit{flow max}
are self-explanatory and were set as low numeric values to possibly increase computation speed and avoid overflow.


\begin{table}[]
    \begin{threeparttable}
    \caption{Parameters used to generate testing scenarios}
    \begin{tabular}{|c|c|c|c|c|c|c|c|}
        \hline
        &
        \begin{tabular}[c]{@{}c@{}}
            layout\\ area ratio
        \end{tabular} &
        \begin{tabular}[c]{@{}c@{}}
            max painting\\ width
        \end{tabular} &
        \begin{tabular}[c]{@{}c@{}}
            max painting\\ height
        \end{tabular} &
        \begin{tabular}[c]{@{}c@{}}
            max painting\\ ratio
        \end{tabular} &
        \begin{tabular}[c]{@{}c@{}}
            flow\\ min
        \end{tabular} &
        \begin{tabular}[c]{@{}c@{}}
            flow\\ max
        \end{tabular} &
        \begin{tabular}[c]{@{}c@{}}
            eval\\ function
        \end{tabular} \\ \hline
        random        & 1.2 & 10 & 10 & 3 & 0 & 4 & $f(x,y) = x+y$ \\ \hline
        clustering    & 1.2 & 10 & 10 & 3 & - & - & $f(x,y) = 0$   \\ \hline
        packing       & 1   & 10 & 10 & 3 & 0 & 4 & $f(x,y) = 0$   \\ \hline
        \begin{tabular}[c]{@{}c@{}}
            biased\\ clustering
        \end{tabular} & 1.3 & 10 & 10 & 3 & - & - & -              \\ \hline
    \end{tabular}
    \begin{tablenotes}
        \small
        \item Left-out values marked with - are discussed later in the text.
    \end{tablenotes}
    \end{threeparttable}
    \label{tab:scenarious-params}
\end{table}

\subsection{Instances}
There are seven instances in total with their parameters described in table~\ref{tab:instances}.
\todo{describe eval function at biased clustering}.
Visualization of a flow between paintings can be seen in two instances in figure~\ref{fig:instance-flow}.
Flow, in other instances, follows the same generation pattern determined by the scenario
– random flow for random and packing scenarios and non-zero flow only between
the same group of paintings in a (biased)-clustering scenario.


\begin{table}[]
    \caption{Parameters of generated instances}
    \begin{tabular}{|c|c|c|c|c|}
        \hline
        instance name &
        \begin{tabular}[c]{@{}c@{}}
            painting\\ count
        \end{tabular} &
        \begin{tabular}[c]{@{}c@{}}
            layout\\ width x height
        \end{tabular} &
        scenario &
        description \\ \hline
        random\_10  & 10 & 24 x 19 & random  & \\ \hline
        random\_20  & 20 & 31 x 25 & random  & \\ \hline
        packing\_10 & 10 & 19 x 15 & packing & \\ \hline
        packing\_20 & 20 & 33 x 26 & packing & \\ \hline
        cluster\_3\_6 & 18 & 30 x 25 & clustering & \begin{tabular}[c]{@{}c@{}}
                                                        3 clusters,\\ 6 paintings each
        \end{tabular} \\ \hline
        cluster\_4\_5 & 20 & 34 x 27 & clustering & \begin{tabular}[c]{@{}c@{}}
                                                        4 clusters,\\ 5 paintings each
        \end{tabular} \\ \hline
        biased\_sparse\_cluster\_3\_5 &
        15 &
        29 x 23 &
        biased clustering &
        \begin{tabular}[c]{@{}c@{}}
            3 clusters,\\ 5 paintings each
        \end{tabular} \\ \hline
    \end{tabular}
    \label{tab:instances}
\end{table}



\newpage


\section{Hyper-parameters}\label{sec:hyper-parameters}

Proposed genetic algorithm~\ref{alg:genetic} has eight hyperparameters.
They are described in table~\ref{tab:hyperparameters-description} and used in listing~\ref{lst:computation-submission-dataset}.
The rest of the sections discuss these hyperparameters further and tries to find their reasonable values.

Two instances are chosen for hyperparameter testing – random\_10 and random\_20.
The reason is that they are not biased towards any preferred solution, e.g., forming clusters.
Thus, insights into the proposed solution can be gained.
However, fine-tuning the hyperparameters to the specific instance or scenario is recommended but only sometimes computationally feasible.

Hyperparameter testing is performed by changing only the hyperparameter under test.
Hyperparameters not under test are identical to the values in listing~\ref{lst:computation-submission-dataset}.
The exceptions are penalization constants $\lambda, \gamma$ (eq.~\ref{eq:objective}), and \verb|populationSize|.
Penalization constants are set to the length of the layout diagonal (see~\ref{subsec:overlapping-penalization-constant}).
Population size is set to $50N$, where $N$ is the size of the instance.
The reason for choosing such parameters as base parameters for testing
is preliminary results (not presented in this thesis), which proved correct in many cases.

Lastly, to achieve the statistical significance of the results presented in this section, each computation\footnotemark[1] is submitted five times with a different random seed.
Presented values are thus an average from five samples.

\footnotetext[1]{
    Copmutations were run on a notebook with Fedora 36, 16GB RAM, AMD Ryzen 7 PRO 4750U 8 x 1.7 - 4.1 GHz, Renoir PRO (Zen 2).
}


\begin{table}[h!]
    \caption[Hyperparameters of the genetic algorithm]{Hyperparameters of the genetic algorithm~\ref{alg:genetic}}
    \label{tab:hyperparameters-description}
    \begin{tabular}{ll}
        \hline
        \textbf{Hyperparameter}   & \textbf{Description}                                       \\ \hline
        \verb|maxNumberOfIter|    & maximum number of iterations                               \\ \hline
        \verb|populationSize|     & population size                                            \\ \hline
        \verb|maximumWildCardCount| & \begin{tabular}[c]{@{}l@{}}
                                          limit on the maximum number of $*$ cut types\\ produced by $OR_{prob}$ decoding
        \end{tabular} \\ \hline
        \verb|orientationWeights| & penalization vector $P$ from eq.~\ref{eq:crossover-orprob} \\ \hline
        \verb|populationDivisionCounts| & \begin{tabular}[c]{@{}l@{}}
                                              reproductive plan ratios\\ (right part of fig.~\ref{fig:population-schema})
        \end{tabular} \\ \hline
        \verb|initialPopulationDivisionCounts| &
        \begin{tabular}[c]{@{}l@{}}
            initial population ratios\\ (left part of fig.~\ref{fig:population-schema})
        \end{tabular} \\ \hline
        \verb|overlappingPenalizationConstant| &
        \begin{tabular}[c]{@{}l@{}}
            overlapping paintings penalization\\ constant $\lambda$ from eq.~\ref{eq:objective}
        \end{tabular} \\ \hline
        \verb|outsideOfAllocatedAreaPenalizationConstant| &
        \begin{tabular}[c]{@{}l@{}}
            outside of allocated area penalization\\ constant $\gamma$ from eq.~\ref{eq:objective}
        \end{tabular} \\ \hline
    \end{tabular}
\end{table}

\newpage
\subsection{Max number of iter}\label{subsec:max-number-of-iter}
Hyperparameter \verb|maxNumberOfIter| determines the number of iterations in the genetic algorithm~\ref{alg:genetic}.

Results for two random instances are in figure~\ref{fig:hyperparameters-max-number-of-iter}.
We can see the initial decrease of the average population objective for both random instances.
Above iteration 300, the initial decreasing trend stops for the random\_10 instance, and for the random\_20 instance, the decrease becomes very slow.

The conclusion is that at least 300 iterations are needed before the average population objective
stops decreasing rapidly.

\subsection{Population size}\label{subsec:population-size}

Hyperparameter \verb|populationSize| is calculated as $\kappa N$, where $\kappa$ is \definice{population scaling factor}
and $N$ is instance size.
It determines the population size that is linear to the instance size.

Results for two random instances are in figure~\ref{fig:hyperparameters-population-size}.
We can see that scaling factor $10$ does not allow
the population objective average to decrease to the levels comparable to scaling factors $50$ and $100$.
It might imply that the scaling factor $10$ cannot represent knowledge gathered over time
in the genetic algorithm or that more iterations are needed.

The conclusion is that using scaling factor between $50$ and $100$ is sufficient, with bias towards $100$
for obtaining better average objective performance.
However, increasing the scaling factor leads to slower computation speed as every population contains
more individuals for which reproductive plan must be computed.

\subsection{Maximum wildcard count}\label{subsec:maximum-wild-card-count}
Hyperparameter \verb|maximumWildCardCount| limits the maximum number of $*$ cut types produced by $OR_{prob}$ decoding (subsec.~\ref{subsec:individual-decoding}).
Keeping this hyperparameter low or even setting it to zero is recommended.
The reason is that if it is high, computation time increases as $*$ spreads in the population.
For example, consider a decoded individual whose orientations $OR$ are solely composed of~$*$ cut types.
Then, the  individual decodes to $2^{|OR|}$ resolved slicing trees as seen in figure~\ref{fig:layout-construction-steps}.

Results for random\_10 instance are in figure~\ref{fig:hyperparameters-maximum-wild-card-count}.
The top sub-figure shows that the average population objective does not differ significantly for any limit on the wildcard cut type.
However, there is a slight advantage for the maximum wildcard count equal to one.
It might be caused by using wildcard penalization $0.5$ (listing~\ref{lst:computation-submission-dataset})
that does not allow the spread of the wildcard in the population.

On the bottom sub-figure, we can see the computation speed as the limit on the wildcard cut type increases.
It grows linearly up to the maximum wildcard count of eight, and then the increase stops.
It might be because the wildcard penalization $0.5$ does not
allow the wildcard cut type to spread over the maximum wildcard count of eight.
Another reason might be computational anomalies caused by high-memory consumption as the maximum wildcard count increases.

The conclusion tested on random\_10 instance is that (a) the maximum wildcard count for wildcard penalization $0.5$
performs similarly for all values, with a slight performance gain if using a maximum wildcard count equal to one,
and (b) computation time is linear with increasing maximum wildcard count and wildcard penalization $0.5$.

\newpage
\subsection{Orientation weights}\label{subsec:orientation-weights}
Hyperparameter \verb|orientationWeights| is the orientation penalization vector $P$ from crossover eq.~\ref{eq:crossover-orprob}.
It determines the bias towards the type of cut ($H$, $V$, $*$, see~\ref{subsec:individual-decoding}).

Only penalization for the wildcard cut type $*$ is tested, as there is no need to penalize or have a preference for $H$ or $V$ cut types.
Also, recall that the hyperparameter \verb|maximumWildCardCount| is set to one during testing, as described at the beginning of the section and showed in listing~\ref{lst:computation-submission-dataset}.

Performance results for two random instances are in figure~\ref{fig:hyperparameters-orientation-weights}.
We can see that for random\_10 instance,
weight does not significantly influence the average population objective, and after iteration 250, differences become negligible.
However, for random\_20 instance,
weight one (no penalization) has a faster-decreasing trend and produces a population with a better average population objective.

The reason for better average performance at larger instance with weight one (no penalization) might
be that search space increases exponentially (there exists at least $2^{N-1}$ different unresolved slicing trees, for instance of size $N$),
and the introduction of wildcard cut type $*$ starts to manifest itself at larger instances.

Results for the average number of wildcard cut types $*$ at the best individual before decoding at each iteration are in
figure~\ref{fig:hyperparameters-orientation-weights-wildcard-cut-type-spread}.
We can see that for the random\_10 instance, weights below one have less than one wildcard.
However, there are between three and four wildcards for a weight equal to one (no penalization).
Similar can be seen for random\_20 instance.

The reason why there are wildcards present in the figure~\ref{fig:hyperparameters-orientation-weights-wildcard-cut-type-spread}
even for weight equal to zero (maximal penalization) is that wildcard can be introduced to a chromosome by mutation or injection of random individuals (see reproductive plan~\ref{subsec:reproductive-plan}).
Described penalization only applies to the crossover.

The conclusion from figure~\ref{fig:hyperparameters-orientation-weights} is that (a) smaller instances, such as random\_10, do not benefit from the introduction of wildcard cut type $*$, and (b) bigger instances, such as random\_20, benefit from no wildcard penalization by having a faster-decreasing trend and producing a better average population.

The conclusion from figure~\ref{fig:hyperparameters-orientation-weights-wildcard-cut-type-spread} is that if we want to be certain that wildcard cut type $*$ is contained in the best individual at each iteration,
wildcard orientation weight must be set close to one.

\subsection{Population division counts}\label{subsec:population-division-counts}
Hyperparameter \verb|populationDivisionCounts| configures ratios in the reproductive plan (subsec.~\ref{subsec:reproductive-plan}).
It influences how the next generation is created in the genetic algorithm~\ref{alg:genetic}
by setting (a) how many elite individuals are copied, (b) how many children are created using a crossover operator,
(c) how many mutants are created using a mutation operator, (d) how many tournament winners are included, and (e) how many random individuals are injected.

Performance results for two random instances are in figure~\ref{fig:hyperparameters-population-division-counts}.
We can see that results do not differ for the smaller or larger instance.
The best reproductive plan strategy is achieved without changing population division hyperparameters from listing~\ref{lst:computation-submission-dataset}.
That means elitism and randomly injected individuals are present.
Additionally, the elitism strategy is the root cause of good performance,
as removing randomly generated individuals does not significantly improve performance.

The reason why the use of elitism is important to obtain good results might be the implementation of the crossover operator (subsec.~\ref{subsec:crossover}).
Crossover adds weights $w_A$ and $w_B$ to each parent based on their objective value (lower objective value achieves bigger weight).
It means that if $w_A \gg w_B$, the offspring is a sample from the search space close to the parent $A$ or nearly identical to $A$.
On the other hand, if the weights are similar and each parent represents a different (sub)-optimal solution,
the transfer of information does not happen, and the offspring is no better than a randomly generated individual.
Elitism avoids crossover by directly copying the best individuals without any modification.
It leads to keeping track of multiple (sub)-optimal solutions simultaneously.
In addition, each time a crossover is applied, one parent is selected from the elite pool,
which supports intensification.

On the other hand, removing elitism produces the worst results.
The reason might be the inability to keep track of competing (sub)-optimal and the fast spread of the first
\definice{macho individual}, an individual that performs significantly better than everyone else.
Without elitism, the most significant way to keep track of found (sub)-optimal solutions is through crossover.
Crossover chooses parents randomly from the average pool, which does not give any constraint on the weights $w_A$ and $w_B$.
Additionally, as mentioned in the previous paragraph, similar weights for two different parents produce no better offspring than randomly generated individuals.
It means that most offspring do not perform well.
On the other hand, as soon as the macho individual appears as the parent in the crossover, the offspring is effectively a copy of a macho individual.
Then, if the macho individual is randomly chosen more than once as a parent in a crossover, it rapidly spreads and takes over the whole population.
The chance of macho-individual taking over the population with elitism is decreased as it deliberately keeps track of multiple competing performant individuals,
making it harder for the macho individual to spread.

The conclusion is that using elitism is essential for obtaining good painting placement solutions
as it can keep track of multiple (sub)-optimal solutions simultaneously.

\subsection{Initial population division counts}\label{subsec:initial-population-division-counts}
Hyperparameter \verb|initialPopulationDivisionCounts| configures generation ratios of the initial population (left part of fig.~\ref{fig:population-schema}).
It consists of randomly generated and greedily generated individuals.

Performance results for two random instances are in figure~\ref{fig:hyperparameters-initial-population-division-counts}.
The different ratios' results do not greatly differ.

The conclusion is that hyperparameter \verb|initialPopulationDivisionCounts| does not significantly affect the obtained results.

\subsection{Overlapping penalization constant}\label{subsec:overlapping-penalization-constant}

Hyperparameter \verb|overlappingPenalizationConstant| is the penalization constant $\lambda$ from eq.~\ref{eq:objective}
used to penalize individuals representing solutions with overlapping paintings.
It is calculated as $\rho D$, where $\rho$ is a diagonal multiple and $D$ is the length of a diagonal in a layout.
Diagonal length $D$ for a layout with width $W$ and height $H$ is $\sqrt{W^2 + H^2}$.

Results of the average overlapping paintings count for the best individual at each iteration are in figure~\ref{fig:hyperparameters-overlapping-penalization-constant}.
We can see that the two lowest diagonal multiples with values $0.5$ and $1.0$ fail to remove most overlapping paintings from the best individuals.
On the other hand, values $2$ and higher can remove overlapping paintings at the smaller instance.
However, in the larger instance, there are still several overlappings present.

The conclusion is that hyperparameter \verb|overlappingPenalizationConstant| should be set at least to
two times the diagonal length of the layout to start penalizing individuals that represent solutions with overlapping paintings.

\subsection{Outside of allocated area penalization constant}\label{subsec:outside-of-allocated-area-penalization-constant}
Hyperparameter~\verb|outsideOfAllocatedAreaPenalizationConstant| is the penalization constant $\gamma$ from eq.~\ref{eq:objective} used to penalize
individuals representing solutions with paintings that are placed outside of their allocated area (see fig.~\ref{fig:allocated-area} and subsec.~\ref{subsec:placement-heuristic}).
It is calculated identically as the overlapping penalization constant (subsec.~\ref{subsec:overlapping-penalization-constant}).
That is, as $\rho D$, where $\rho$ is a diagonal multiple, and $D$ is the length of a diagonal in a layout.

Results of the percentage of paintings placed outside the allocated area for two random instances are in table~\ref{tab:hyperparameters-outside-of-allocated-area-penalization-constant}
(it is presented using a table because the values do not significantly change throughout the iterations).
We can see that the outside of allocated area penalization constant cannot force the creation of the allocated space that would fit the vast majority of the paintings.
Interestingly, there are a few percentage points drops in favor of the larger instance.
The reason might be that it has more degrees of freedom, i.e.,  more possibilities to create a cut.
It can thus create more fitting slicing layouts.

However, the failure of the outside of allocated area penalization constant to force
the creation of sufficient allocated space in most cases is not that important.
The reason is that the placing heuristic (subsec.~\ref{subsec:placement-heuristic})
tries to place the painting at several placement points in the allocated area (see fig.~\ref{fig:allocated-area}).
It can thus balance the few paintings that have been allocated sufficient area to avoid
and account for other parts of the objective function, e.g., overlapping paintings.

The conclusion is that the hyperparameter \verb|outsideOfAllocatedAreaPenalizationConstant| is the least
important and might be left out by setting it to zero to save computation time.


\begin{table}[h!]
    \caption{Percentage of paintings placed outside allocated area}
    \label{tab:hyperparameters-outside-of-allocated-area-penalization-constant}
    \shorthandoff{-}
    \begin{threeparttable}
        \begin{tabular}{lll}
            \cline{1-3}
            & \multicolumn{2}{c}{\textbf{Instance}} \\ \cline{2-3}
            \textbf{\begin{tabular}[c]{@{}l@{}}
                        Diagonal\\ multiple
            \end{tabular}} & \multicolumn{1}{c}{random\_10} & \multicolumn{1}{c}{random\_20} \\ \hline
            0              & 99.6                           & 93.3                           \\ \hline
            0.5            & 99.6                           & 94.1                           \\ \hline
            1              & 98                             & 92.5                           \\ \hline
            2              & 97.4                           & 95.9                           \\ \hline
            3              & 98.9                           & 91.7                           \\ \hline
            4              & 99.8                           & 94.9                           \\ \hline
            10             & 99.7                           & 96.1                           \\ \hline
            50             & 98.6                           & 97.8                           \\ \hline
        \end{tabular}
        \begin{tablenotes}
            \small
            \item Percentage is averaged over all iterations for the best individual at each iteration.
        \end{tablenotes}
    \end{threeparttable}
    \shorthandon{-}
\end{table}

\begin{figure}[h!]
    \centering
    \subfloat{\includegraphics[width=0.8\textwidth]{hyperparameters/max_number_of_iter_random_10}\label{subfig:hyperparameters-max-number-of-iter-random-10}}

    \subfloat{\includegraphics[width=0.8\textwidth]{hyperparameters/max_number_of_iter_random_20}\label{subfig:hyperparameters-max-number-of-iter-random-20}}
    \caption[Testing maximum number of iterations]
    {Testing maximum number of iterations at two random instances.}
    \label{fig:hyperparameters-max-number-of-iter}%
\end{figure}

\begin{figure}[h!]
    \centering
    \subfloat{\includegraphics[width=0.9\textwidth]{hyperparameters/population_size_random_10}\label{subfig:hyperparameters-population-size-random-10}}

    \subfloat{\includegraphics[width=0.9\textwidth]{hyperparameters/population_size_random_20}\label{subfig:hyperparameters-population-size-random-20}}
    \caption[Testing population scaling factor]
    {Testing population scaling factor at two random instances.
    The population size is $\kappa N$ for population scaling factor $\kappa$ and instance of size $N$. }
    \label{fig:hyperparameters-population-size}%
\end{figure}

\begin{figure}[h!]
    \centering
    \subfloat{\includegraphics[width=1.1\textwidth]{hyperparameters/maximum_wild_card_count_performance_random_10}\label{subfig:hyperparameters-maximum-wild-card-count-performance}}

    \subfloat{\includegraphics[width=0.8\textwidth]{hyperparameters/maximum_wild_card_count_computation_time_random_10}\label{subfig:hyperparameters-maximum-wild-card-count-computation-speed}}
    \caption[Testing maximum wildcard count]
    {Testing increasing maximum wildcard count. Performance (top) and computation speed (bottom) are showed.}
    \label{fig:hyperparameters-maximum-wild-card-count}%
\end{figure}


\begin{figure}[h!]
    \centering
    \subfloat{\includegraphics[width=1\textwidth]{hyperparameters/orientation_weights_random_10}\label{subfig:hyperparameters-orientation-weights-random-10}}

    \subfloat{\includegraphics[width=1\textwidth]{hyperparameters/orientation_weights_random_20}\label{subfig:hyperparameters-orientation-weights-random-20}}
    \cprotect\caption[Testing orientation weights performance]
    {Testing performance of orientation weight for a wildcard cut type $*$ at two random instances.
    Hyperparameter \verb|maximumWildCardCount| is 1.}
    \label{fig:hyperparameters-orientation-weights}%
\end{figure}

\begin{figure}[h!]
    \centering
    \subfloat{\includegraphics[width=0.9\textwidth]{hyperparameters/orientation_weights_wildcard_cut_type_spread_random_10}\label{subfig:hyperparameters-orientation-weights-wildcard-cut-type-spread-random-10}}

    \subfloat{\includegraphics[width=0.9\textwidth]{hyperparameters/orientation_weights_wildcard_cut_type_spread_random_20}\label{subfig:hyperparameters-orientation-weights-wildcard-cut-type-spread-random-20}}
    \cprotect\caption[Testing wildcard spread]
    {Testing wildcard cut type $*$ spread at two random instances.
    Each iteration shows an average count of wildcard cut type $*$ at best individual before decoding for different values of wildcard orientation weights.
    Hyperparameter \verb|maximumWildCardCount| is 1.}
    \label{fig:hyperparameters-orientation-weights-wildcard-cut-type-spread}%
\end{figure}

\begin{figure}[h!]
    \centering
    \subfloat{\includegraphics[width=1.05\textwidth]{hyperparameters/population_division_counts_random_10}\label{subfig:hyperparameters-population-division-counts-random-10}}

    \subfloat{\includegraphics[width=1.05\textwidth]{hyperparameters/population_division_counts_random_20}\label{subfig:hyperparameters-population-division-counts-random-20}}
    \caption[Testing population division counts]
    {Testing population division counts at two random instances. Four variants are displayed.
    The first does not use elitism.
    The second does not inject random individuals.
    The third combines the first and second, and the last does not change the population division counts as described in listing~\ref{lst:computation-submission-dataset}.}
    \label{fig:hyperparameters-population-division-counts}%
\end{figure}

\begin{figure}[h!]
    \centering
    \subfloat{\includegraphics[width=0.95\textwidth]{hyperparameters/initial_population_division_counts_random_10}\label{subfig:hyperparameters-initial-population-division-counts-random-10}}

    \subfloat{\includegraphics[width=0.95\textwidth]{hyperparameters/initial_population_division_counts_random_20}\label{subfig:hyperparameters-initial-population-division-counts-random-20}}
    \caption[Testing initial population division counts]
    {Testing initial population division counts at two random instances.
    The initial population consists of randomly and greedily generated individuals (left part of fig.~\ref{fig:population-schema}).}
    \label{fig:hyperparameters-initial-population-division-counts}%
\end{figure}

\begin{figure}[h!]
    \centering
    \subfloat{\includegraphics[width=0.95\textwidth]{hyperparameters/overlapping_penalization_constant_random_10}\label{subfig:hyperparameters-overlapping-penalization-constant-random-10}}

    \subfloat{\includegraphics[width=0.95\textwidth]{hyperparameters/overlapping_penalization_constant_random_20}\label{subfig:hyperparameters-overlapping-penalization-constant-random-20}}
    \caption[Testing overlapping penalization constant]
    {Testing overlapping penalization constant $\lambda$ (eq.~\ref{eq:objective}) at two random instances.
    It is calculated as $\rho D$, where $\rho$ is a diagonal multiple, and $D$ is the length of a diagonal in a layout.
    Graphs show the average overlapping paintings count for the best individual at each iteration.}
    \label{fig:hyperparameters-overlapping-penalization-constant}%
\end{figure}



\clearpage% Flush earlier floats (otherwise order might not be correct)
\newpage


\section{Results}\label{sec:results}
This section presents a painting placement solution to the seven painting placement instances described in table~\ref{tab:instances}.
Hyperparameter values used to obtain all results in this section are in table~\ref{tab:hyperparameters-values}.
They are selected based on the hyperparameter testing from section~\ref{sec:hyper-parameters}.
Best achieved objective for each instance is in table~\ref{TODO}.

The only difference in hyperparameter selection is that \verb|maxNumberOfIter| is set to 500 instead of recommended range \numrange{100}{150} to possibly find better painting placement solution.
Other hyperparameters are set to their recommended values from section~\ref{sec:hyper-parameters}.
Hyperparameter \verb|populationSize| is set as the middle of the recommended range \numrange{50}{100}.
Penalization for orientation weight is removed by setting \verb|orientationWeights| to $\langle 1,1,1\rangle$.


\begin{table}[h!]
    \caption{Hyperparameter values used to obtain results}
    \label{tab:hyperparameters-values}
    \begin{threeparttable}
        \begin{tabular}{ll}
            \hline
            \textbf{Hyperparameter}                           & \textbf{Value}             \\ \hline
            \verb|maxNumberOfIter|                            & 300                        \\ \hline
            \verb|populationSize|                             & 75 times the instance size \\ \hline
            \verb|maximumWildCardCount|                       & 1                          \\ \hline
            \verb|orientationWeights|                         & $\langle 1,1,1 \rangle$    \\ \hline
            \verb|populationDivisionCounts|                   & remove elitism and random  \\ \hline
            \verb|initialPopulationDivisionCounts|            & 0.7 random, 0.3 greedy     \\ \hline
            \verb|overlappingPenalizationConstant| & \begin{tabular}[c]{@{}l@{}}
                                                         4 times the diagonal length\\ of the layout
            \end{tabular} \\ \hline
            \verb|outsideOfAllocatedAreaPenalizationConstant| & 0                          \\ \hline
        \end{tabular}
        \begin{tablenotes}
            \small
            \item Hyperparameter description is in table~\ref{tab:hyperparameters-description}.
        \end{tablenotes}
    \end{threeparttable}
\end{table}

\begin{table}[h!]
    \caption{Statistics of the last iteration}
    \label{tab:statistics}
    \begin{threeparttable}
        \begin{tabular}{lllll}
            \hline
            \textbf{Instance name} &
            \textbf{\begin{tabular}[c]{@{}l@{}}Best obj.\\ value\end{tabular}} &
            \textbf{\begin{tabular}[c]{@{}l@{}}Worst obj.\\ value\end{tabular}} &
            \textbf{\begin{tabular}[c]{@{}l@{}}Obj.\\ mean\end{tabular}} &
            \textbf{\begin{tabular}[c]{@{}l@{}}Standard\\ deviation\end{tabular}} \\ \hline
            random\_10                    & 1106.33 & 2932.9  & 1475.31 & 319.91  \\ \hline
            random\_20                    & 5106.49 & 8617.66 & 5521.33 & 647.47  \\ \hline
            packing\_10                   & 647.47  & 1482.71 & 748.28  & 126.11  \\ \hline
            packing\_20                   & 6607.76 & 10881.3 & 6996.43 & 792.16  \\ \hline
            cluster\_3\_6                 & 4940.93 & 8924.68 & 5258.09 & 517.73  \\ \hline
            cluster\_4\_5                 & 5080.58 & 9664.94 & 5592.13 & 695.56  \\ \hline
            biased\_sparse\_cluster\_3\_5 & TODO    & TODO    & TODO    & TODO    \\ \hline
            london\_gallery\_wall         & 2802.73 & 8487.18 & 3471.79 & 3471.79 \\ \hline
        \end{tabular}
        \begin{tablenotes}
            \small
            \item Instance description is in table~\ref{tab:instances}.
        \end{tablenotes}
    \end{threeparttable}
\end{table}

\begin{figure}[h!]
    \includegraphics[width=0.8\textwidth, center]{visualizations/visualization_random_10}
    \caption
    {TODO overlappings=1}
    \label{fig:results:visualization-random-10}
\end{figure}

\begin{figure}[h!]
    \includegraphics[width=0.8\textwidth, center]{visualizations/visualization_random_20}
    \caption
    {TODO overlappings=2}
    \label{fig:results:visualization-random-20}
\end{figure}

\begin{figure}[h!]
    \includegraphics[width=0.8\textwidth, center]{visualizations/visualization_packing_10}
    \caption
    {TODO overlappings=0}
    \label{fig:results:visualization-packing-10}
\end{figure}

\begin{figure}[h!]
    \includegraphics[width=0.8\textwidth, center]{visualizations/visualization_packing_20}
    \caption
    {TODO overlappings=8}
    \label{fig:results:visualization-packing-20}
\end{figure}

\begin{figure}[h!]
    \includegraphics[width=0.8\textwidth, center]{visualizations/visualization_cluster_3_6}
    \caption
    {TODO overlappings=4}
    \label{fig:results:visualization-cluster-3-6}
\end{figure}

\begin{figure}[h!]
    \includegraphics[width=0.8\textwidth, center]{visualizations/visualization_cluster_4_5}
    \caption
    {TODO overlappings=8}
    \label{fig:results:visualization-cluster-4-5}
\end{figure}

\begin{figure}[h!]
%    \includegraphics[width=0.8\textwidth, center]{visualizations/visualization_biased_sparse_cluster}
    \includegraphics[width=0.8\textwidth, center]{placeholder}
    \caption
    {TODO biased sparse cluster}
    \label{fig:results:visualization-biased-sparse-cluster}
\end{figure}


\begin{figure}[h!]
    \includegraphics[width=0.8\textwidth, center]{visualizations/visualization_london_gallery_wall}
    \caption
    {TODO overlappings=0}
    \label{fig:results:visualization-london-gallery-wall}
\end{figure}



\newpage
\section{Implementation}\label{sec:implementation}

The proposed implementation of a genetic approach is written in Java 11
using a Play Framework v2.8\footnotemark[1], a web framework for Java and Scala.

Implementation behaves like a computation server to which
a user can submit a computation.
Then, the server asynchronously starts the submitted computation and returns an identifier of the computation.
It means that multiple computations can be submitted without blocking the user.
The user can then check the computation state using the returned identifier.

To start the computation server, locate the directory containing a file \verb|build.sbt| in the attached medium (see appendix~\ref{ch:contents-of-the-attached-medium}).
Then, run the following command in that directory (Java~11, SBT\footnotemark[3], and Scala must be installed).

\begin{listing}[h!]
    \begin{cminted}[autogobble,breaklines=true]{shell}
        $ sbt run
    \end{cminted}
    \caption[Starting a computation server]
    {Starting a computation server.}
    \label{lst:sbt-run}
\end{listing}

Command in listing~\ref{lst:sbt-run} uses \verb|sbt|\footnotemark[3] with \verb|run| argument to start the computation server.
By default, the computation server accept requests on \verb|localhost:9000|.
An example of submitting a computation with a predefined instance name to the computation server is in listing~\ref{lst:computation-submission-dataset}.
An example of a successful computation submission response is in listing~\ref{lst:computation-response-success}.

\begin{listing}[h!]
%    \centering
    \begin{cminted}[autogobble,breaklines=true]{json}
    {
        "id":"random_10_9B5F8",
        "outputDirectory":"./out/088_random_10_9B5F8"
    }
    \end{cminted}
    \caption[Successful computation submission response]
    {Successful computation submission response.}
    \label{lst:computation-response-success}
\end{listing}

The computation server also validates input before starting the computation.
For example, if misspelling the instance name, the response by the computation server can be seen in~\ref{lst:computation-response-fail}.

\begin{listing}[h!]
    \centering
    \begin{cminted}[autogobble,breaklines=true]{json}
    {
        "message":"Entity [DatasetDto] with identifier [randomm_10] was not found."
    }
    \end{cminted}
    \caption[Unsuccessful computation submission response]
    {Unsuccessful computation submission response.}
    \label{lst:computation-response-fail}
\end{listing}

Lastly, there is also an option not to specify the instance name in the submission, as seen
in listing~\ref{lst:computation-submission-dataset} where the instance name is random\_10.
In that case, a user has to specify the instance manually in the request – layout width and height,
paintings together with their flow and evaluation function $\pi$ (eq.~\ref{eq:objective}), in the format that is accepted by
mXparser\footnotemark[4].
An example of submission without specifying the instance name is in the appendix in listing~\ref{lst:computation-submission-manual}.

\begin{listing}[h!]
\centering
\begin{minted}[autogobble,breaklines=true]{shell}
$ curl --location 'localhost:9000/compute/dataset' \
--header 'Content-Type: application/json' \
--data '{
  "datasetName": "random_10",
  "gaParameters": {
    "maxNumberOfIter": 300,
    "populationSize": 500,
    "maximumWildCardCount": 1,
    "orientationWeights": [
      1,
      1,
      0.5
    ],
    "geneticAlgorithm": "simpleGa",
    "mate": "normalizedProbabilityVectorSum",
    "mutate": "flipOnePartAtRandom",
    "select": "tournament",
    "objective": "simple",
    "evaluator": "ga",
    "placingHeuristics": "corner",
    "populationDivisionCounts": {
      "elite": 0.2,
      "average": 0.6,
      "worst": 0.2,
      "children": 0.3,
      "mutant": 0.2,
      "winner": 0.2,
      "random": 0.1
    },
    "initialPopulationDivisionCounts": {
      "random": 0.7,
      "greedy": 0.3
    }
  },
  "objectiveParameters": {
    "name": "simple",
    "params": {
      "overlappingPenalizationConstant": 30.61,
      "outsideOfAllocatedAreaPenalizationConstant": 30.61
    }
  }
}'
\end{minted}
\cprotect\caption[Example of computation submission with instance name]
{Example of computation submission of random\_10 instance using \verb|curl|\footnotemark[2] to a computation server running on \verb|localhost:9000|.}
\label{lst:computation-submission-dataset}
\end{listing}

\footnotetext[1]{\url{https://www.playframework.com/documentation/2.8.x/Home}}
\footnotetext[2]{\url{https://curl.se/}}
\footnotetext[3]{\url{https://www.scala-sbt.org/}}
\footnotetext[4]{\url{https://mathparser.org/}}


\chapter{Discussion}\label{ch:discussion}

This chapter discusses the obtained results and further possible improvements or extensions for the proposed solution.
First, it describes several ways the proposed solution can be improved or extended in section~\ref{sec:implementation}.


\section{Improvements and extensions}\label{sec:improvements}

\subsection{Free space}\label{subsec:free-space}

The extension of the proposed solution is to take a different approach to free space.
\definice{Free space} can be defined as a part of the painting placement solution where the painting is not placed.
For example, free space is important in the FLP problem as there is a need for an aisle between the facilities
through which the material transportation takes place~\cite{scholzExtensionsSTaTSPractical2010}.

In the proposed solution, two main parts influence where the free space is created.
It is (1) the placing heuristic and (2) the evaluation function $\pi$, see eq.~\ref{eq:objective}.
Placing heuristic works locally, i.e., only in the allocated area for the painting, and the evaluation
function, although it might be used to define arbitrary free space shape, is not a constraint but a penalization.
Thus, it does not guarantee that the painting placement solution will create a solution with the desired free space.

One possible approach to guarantee free space is the introduction of dummy paintings.
These dummy paintings can be injected during the slicing tree construction.
The resulting painting placement solution will thus, among the paintings, contain free space occupied by dummy paintings.

An example of dummy painting injection can be seen in figure~\ref{fig:dummy-painting}.
The aisle is created between paintings 1,2, and 3 by adding a vertical cut $V$ to the tree.


\begin{figure}[h!]
    \includegraphics[height=0.33\textheight, center]{dummy_painting}
    \caption{Example of dummy painting injection.}
    \label{fig:dummy-painting}
\end{figure}

\subsection{Non-rectangular layouts}\label{subsec:non-rectangular-layouts}

Another extension to the proposed solution is adding the ability to work with layouts that are not rectangular.
It can be solved using the dummy paintings described in subsection~\ref{subsec:free-space}.
These dummy paintings are as small as possible and placed over the parts of the layout that are not rectangular.
By placing these dummy facilities, the layout becomes rectangular.
A similar approach is used in~\cite{scholzExtensionsSTaTSPractical2010} to modify a slicing tree to solve FLP.

An example of using dummy paintings to work with a non-rectangular layout is in figure~\ref{fig:non-rectangular-layout}.
There are two irregularities in both corners of the layout.
Two dummy paintings are injected into the slicing tree using a dummy painting injection to fill the irregularities.

\begin{figure}[h!]
    \includegraphics[width=0.8\textwidth, center]{non-rectangular-layout}
    \caption[Example of working with non-rectangular layout.]{Example of working with non-rectangular layout. The allocated area is marked using a dotted line.
    Dummy paintings that fill the irregularities in both corners are A and B.}
    \label{fig:non-rectangular-layout}
\end{figure}

\subsection{Non-rectangular paintings}\label{subsec:non-rectangular-paintings}

Not only can the layout be non-rectangular, but also paintings can have a non-rectangular shape.
Thus, another extension is to allow painting shapes that are not rectangular.
This problem can be easily solved by representing a non-rectangular painting
as the smallest possible rectangle to which the painting fits.
By using this approach, the proposed implementation can work with non-rectangular paintings.
One possible drawback is that the painting placement solution might become more sparse,
i.e., containing too much free space.
However, this can be solved using a post-optimization heuristic that tries to reduce a free space
of a painting placement solution.

\subsection{Placing heuristic}\label{subsec:placing-heuristic}

Instead of using the placing heuristic described in~\ref{fig:corner-placing-heuristic}, a different one can be used.
One candidate is a heuristic that, instead of trying to place painting in the corners of the allocated area,
tries to place the painting at all possible placement points, e.g., to place the painting in the middle of the allocated area.
However, using this solution might be computationally expensive.
On the other hand, a heuristic that only tries the bottom-left of the allocated area as a placement point can be much faster but
might not produce good results.

Another approach is calculating suboptimal or optimal placement point inside the allocated area.
Then, move the painting as close as possible to that point.
A similar approach is used in~\cite{goncalvesBiasedRandomkeyGenetic2015} for solving FLP.
They move the facility centroid as close as possible to the calculated unconstrained optimum without leaving the
boundaries where the facility can be placed.

\subsection{Post-optimization}\label{subsec:post-optimization}

One interesting idea is to introduce post-optimization.
This is a process that takes the result, in this case painting placement solution,
and tries to improve it.
For example, if there is not enough free space between a paintings, they can be moved by the post-optimization process.
Another example would be if the goal is to create the most compact layout.
Then, solution can be the compaction operation proposed in~\cite{laiSlicingTreeComplete2001}
which tries to reduce free space between paintings as much as possible.

\subsection{Extension to other problems}\label{subsec:extension-to-other-problems}

This thesis solves the painting placement problem.
The novel genetic approach uses stochastic vector representation with a crossover that adds stochastic vectors together.
This approach can be applied to other optimization problems.
The suitable ones are the ones that optimize some permutation of elements.
For example, elements can be rectangles, and their permutation is used
as an input to a placing heuristic.

One concrete example is 2D-KP problem with rectangular pieces~\cite{bortfeldtGeneticAlgorithmTwodimensional2009},
where the objective is to place as many rectangles in a container as possible,
minimizing unused space.
The solution proposed in this thesis can solve the 2D-KP problem with rectangular pieces by
representing an individual as a stochastic vector that is then decoded
in the same way as the painting sequence random key described in Individual decoding~\ref{subsec:individual-decoding}.
Then, this decoding produces a sequence of rectangles used as an input to BL heuristic~\cite{chazelleBottomnLeftBinPackingHeuristic1983}.
BL heuristic then places all the rectangles inside the container.
After the placing is finished, all rectangles placed outside the container are removed.

\subsection{Future work}\label{subsec:future-work}

There are at least three ideas for the painting placement problem and the proposed implementation that future work might try to address.

\begin{itemize}
    \item Deciding which painting to place.
    \item Multiple walls for painting placement.
    \item Human operator assistance.
\end{itemize}

Deciding which painting to place can happen when the paintings exceed a wall's area.
Thus, some subset of paintings has to be selected and placed.
Also, there can be multiple walls.
Additionally, the walls might have some interactions between them, e.g., placing a painting on the first wall
limits the subset of paintings that can be placed on the second wall.
Lastly, the produced result can be easily visualized.
Thus, it can be used as an input to some human operators that will further modify it.
For example, FLP, see section~\ref{sec:facility-layout-problem}, which creates a layout for multiple facilities.
A human engineer that creates the plan for placing these facilities might use the output of an algorithm as a starting point.
Then, the engineer can modify it by increasing the aisle or changing the position of some facilities.

\chapter{Conclusion}\label{ch:conclusion}

The central part of the thesis was to propose a genetic approach for solving the painting placement problem.
It was accomplished by creating a genetic algorithm with novel individual representation as multiple stochastic vectors
and novel crossover as vector addition, followed by normalization back to the stochastic vector.
Subsequent parts and goals of the thesis were defined in the introduction chapter~\ref{ch:introduction}.
All of them were accomplished and presented in this thesis.

\begin{enumerate}
    \item The first goal was to define a painting placement problem and what a solution is.
    It was defined in terms of a painting placement instance, which consists
    of paintings, flow between paintings, layout, and evaluation function.
    The solution was defined as a sequence of placement points for the paintings.

    \item  The second goal was to create a dataset for the painting placement problem.
    Three scenarios were defined – random, clustering, and packing.
    Multiple instances of the painting placement problem were generated for each scenario.
    Additionally, London National Gallery painting placement instance was created from a figure~\ref{fig:london-wall}.

    \item  The thesis's third and central goal was to propose and implement a genetic approach for solving the painting placement problem.
    As described above, an individual is represented as multiple stochastic vectors in the novel genetic approach.
    Then, the crossover is implemented as vector addition, followed by normalization back to the stochastic vector.
    These vectors decode into one slicing tree, which recursively divides the space or wall where the paintings are placed.
    The second novel approach was to add a wildcard symbol $*$ to the possible values contained in the internal node of a slicing tree.
    Wildcard symbol $*$ can take up any value – $H$ for horizontal cut and $V$ for vertical cut.

    \item  The fourth goal was to evaluate the performance of the proposed genetic approach.
    It was achieved by creating a computational server to which a painting placement
    instance can be submitted and its painting placement solution obtained.
    Each instance in the dataset was submitted multiple times to the computation server to account for the statistical significance of the obtained results.
    These results were then presented and discussed.

    \item  The fifth and last goal was to discuss the results and suggest further improvements, extensions, and future work.
    Suggestions and result discussion were presented in chapter~\ref{ch:discussion}.
    One suggestion was creating an empty space inside the resulting layout by injecting dummy paintings into the slicing tree.
    This injection could be further used to work with non-rectangular layouts.
    Additionally, human operator assistance was mentioned, where the painting placement solution is presented to a human operator
    that further modifies it to his/her needs.

\end{enumerate}

 % include `text.tex' from `text/' subdirectory

    \printbibliography[heading=bibintoc] % print out the BibLaTeX-generated bibliography list

    \appendix\appendixinit % do not remove these two commands

    \chapter{Appendix}\label{ch:appendix}


\begin{listing}[h!]
    \centering
    \begin{cminted}[autogobble,breaklines=true]{shell}
        $ curl --location 'localhost:9000/compute' \
        --header 'Content-Type: application/json' \
        --data '{
            "instanceParameters": {
                "layout": {
                    "width": 30,
                    "height": 20,
                    "evalFunc": "f(x,y) = x+y"
                },
                "paintings": [
                    { "ident": "1", "width": 5, "height": 7 },
                    { "ident": "2", "width": 5, "height": 7 },
                    { "ident": "3", "width": 5, "height": 7 },
                    { "ident": "4", "width": 5, "height": 7 },
                    { "ident": "5", "width": 5, "height": 7 },
                    { "ident": "6", "width": 5, "height": 7 }
                ],
                "paintingsFlow": [
                    { "from": 1, "to": 2, "flow": 3.3 },
                    { "from": 1, "to": 3, "flow": 4.4 },
                    { "from": 1, "to": 5, "flow": 0 }
                ]
            },
            "gaParameters": {
                "maxNumberOfIter": 2,
                "populationSize": 100,
                "maximumWildCardCount": 1,
                "orientationWeights": [ 1, 1, 0.5 ],
                "geneticAlgorithm": "simpleGa",
                "mate": "normalizedProbabilityVectorSum",
                "mutate": "flipOnePartAtRandom",
                "select": "tournament",
                "objective": "simple",
                "evaluator": "ga",
                "placingHeuristics": "corner",
                "populationDivisionCounts": {
                    "elite": 0.2, "average": 0.6, "worst": 0.2,
                    "children": 0.3, "mutant": 0.2, "winner": 0.2,
                    "random": 0.1
                },
                "initialPopulationDivisionCounts": {
                    "random": 0.7, "greedy": 0.3
                }
            },
            "objectiveParameters": {
                "name": "simple",
                "params": {
                    "overlappingPenalizationConstant": 1000,
                    "outsideOfAllocatedAreaPenalizationConstant": 0
                }
            }
        }'
    \end{cminted}
    \cprotect\caption[Example of computation submission without instance.]{Example of computation submission using \verb|curl|\footnotemark[1] without specyfing the instance name.
    Without it, everything has to be entered manually into the request – layout width and height, paintings together with their flow and evaluation function.}

    \label{lst:computation-submission-manual}
\end{listing}

\footnotetext[1]{\url{https://curl.se/}}
 % include `appendix.tex' from `text/' subdirectory

    \backmatter % do not remove this command

    \chapter{Contents of the attached medium}


	\dirtree{%
		.1 README.md\DTcomment{the file with attached medium contents description}.
		.1 impl\DTcomment{the directory with implementation source code}.
		.2 public/datasets\DTcomment{the directory of datasets}.
		.1 notebooks\DTcomment{the directory of Jupyter notebooks for visualization and dataset generation}.
		.1 thesis\DTcomment{the directory with thesis}.
		.2 ctufit-thesis-en.tex\DTcomment{the thesis source in TEX format}.
		.2 ctufit-thesis-en.pdf\DTcomment{the thesis text in PDF format}.
	}
 % include `medium.tex' from `text/' subdirectory

\end{document}
