\section{Coding}\label{sec:coding}

\subsection{Individual representation}\label{subsec:individual-representation}

\begin{figure}[htp]

    \begin{tabular}{lccccccc}
        Painting sequence random key & 0.1 &                     & 0.2 &                     & 0.3 &                     & 0.4 \\
        Slicing order random key     &     & 0.6                 &     & 0.1                 &     & 0.3                 &     \\
        Orientation probabilities    &     & {[}0.3, 0.2, 0.5{]} &     & {[}0.1, 0.2, 0.7{]} &     & {[}0.7, 0.2, 0.1{]} &
    \end{tabular}
    \caption{Example of an individual representation.}
    \label{tab:ind-rep-example}

    \bigskip

    \begin{tabular}{lccccccc}
        Painting sequence random key & $0.1_1$ &                                      & $0.2_2$ &                                      & $0.3_3$ &                                      & $0.4_4$ \\
        Slicing order random key     &         & $0.1_2$                              &         & $0.3_3$                              &         & $0.6_1$                              &         \\
        Orientation probabilities    &         & [$0.3_H$, $0.2_V$, $\mathbf{0.5_*]}$ &         & [$0.1_H$, $0.2_V$, $\mathbf{0.7_*}$] &         & [$\mathbf{0.7_H}$, $0.2_V$, $0.1_*$] &
    \end{tabular}
    \caption{An individual with sorted random keys. A lower index at the random key indicates the ordinal position before sorting – the painting sequence was already sorted, but the slicing order was not. Bold orientation probabilities are the most probable, with a lower index indicating the orientation associated with the probability.}
    \label{tab:ind-rep-sorted}

    \bigskip

    \begin{tabular}{lccccccc}
        Painting sequence & 1 &   & 2 &   & 3 &   & 4 \\
        Slicing order     &   & 2 &   & 3 &   & 1 &   \\
        Orientations      &   & H &   & H &   & * &
    \end{tabular}
    \caption{A decoded individual – this representation is used to create the slicing tree. This representation was created from the previous one by taking lower indexes of sorted random keys and most probable orientation.}
    \label{tab:ind-decoded}

\end{figure}


