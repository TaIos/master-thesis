\chapter{Coding}\label{ch:coding}


\section{Individual representation}\label{subsec:individual-representation}

\begin{figure}[htp]

    \begin{tabular}{lccccccc}
        Painting sequence random key & 0.1 &                     & 0.2 &                     & 0.3 &                     & 0.4 \\
        Slicing order random key     &     & 0.6                 &     & 0.1                 &     & 0.3                 &     \\
        Orientation probabilities    &     & {[}0.3, 0.2, 0.5{]} &     & {[}0.1, 0.2, 0.7{]} &     & {[}0.7, 0.2, 0.1{]} &
    \end{tabular}
    \caption{Example of an individual representation.}
    \label{tab:ind-rep-example}

    \bigskip

    \begin{tabular}{lccccccc}
        Painting sequence random key & $0.1_1$ &                                      & $0.2_2$ &                                      & $0.3_3$ &                                      & $0.4_4$ \\
        Slicing order random key     &         & $0.1_2$                              &         & $0.3_3$                              &         & $0.6_1$                              &         \\
        Orientation probabilities    &         & [$0.3_H$, $0.2_V$, $\mathbf{0.5_*]}$ &         & [$0.1_H$, $0.2_V$, $\mathbf{0.7_*}$] &         & [$\mathbf{0.7_H}$, $0.2_V$, $0.1_*$] &
    \end{tabular}
    \caption{An individual with sorted random keys. A lower index at the random key indicates the ordinal position before sorting – the painting sequence was already sorted, but the slicing order was not. Bold orientation probabilities are the most probable, with a lower index indicating the orientation associated with the probability.}
    \label{tab:ind-rep-sorted}

    \bigskip

    \begin{tabular}{lccccccc}
        Painting sequence & 1 &   & 2 &   & 3 &   & 4 \\
        Slicing order     &   & 2 &   & 3 &   & 1 &   \\
        Orientations      &   & H &   & H &   & * &
    \end{tabular}
    \caption{A decoded individual – this representation is used to create the slicing tree. This representation was created from the previous one by taking lower indexes of sorted random keys and most probable orientation.}
    \label{tab:ind-decoded}

\end{figure}


\section{Operators}\label{sec:operators}
This sections describes operators – actions on the individual representation
seen in section~\ref{subsec:individual-representation}.

We can guide the genetic search in two basic ways – diversification and intensification.
If the diversification is high, the search process is biased towards exploring genomes
that may be vastly different from the one already present in the population.
This tries to prevent the genetic search from getting stuck in a local optimum.
Diversification that is too high would reassemble a random walk, ultimately randomly sampling
from the space of all possible genomes. On the other hand, if the intensification is high,
biased towards exploiting similar genemos takes precedence.

It is crucial to find the correct balance between diversification and intensification.
One way to control the diversification in genetic algorithm is setting
mutation probability – high chance of mutation means higher diversification, low change is the opposite.
Intensification can be similarly controller by the crossover probability.
\ref{TODO}

\subsection{Mutation}\label{subsec:mutation}
Mutation takes place on all 3 parts of a chromosome, as seen in example in table \ref{tab:ind-rep-example}.
Since both random keys vectors and orientation probabilities form a probability distribution vector,
it is easy to perform mutation by replacing a value with a randomly generated one from interval $[0,1]$
and then normalize back to the probability distribution.

For example, consider $$[0.6, 0.1, 0.3]$$ be a painting sequence random key or slicing order random key.
Mutation operator would randomly choose one position in the vector, say the first one, and replace it
with a randomly generated one from interval $[0,1]$. The result may be $$[0.9, 0.1, 0.3]$$.
The mutated vector forms no longer a probability distribution and to obtain the mutated value
it needs to be normalized – finally ending up with $[0.7, 0.07, 0.23]$.

Mutation for the orientation probabilities is the same but with the first additional
step of randomly choosing which of the probability vectors will undergo the aforementioned process.

\subsection{Crossover}\label{subsec:crossover}


