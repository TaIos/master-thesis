\chapter{Conclusion}\label{ch:conclusion}

The central part of the thesis was to propose a genetic approach for solving the painting placement problem.
It was accomplished by creating a genetic algorithm with novel individual representation as multiple stochastic vectors
and novel crossover as vector addition, followed by normalization back to the stochastic vector.
Subsequent parts and goals of the thesis were defined in the introduction chapter~\ref{ch:introduction}.
All of them were accomplished and presented in this thesis.

\begin{enumerate}
    \item The first goal was to define a painting placement problem and what a solution is.
    It was defined in terms of a painting placement instance, which consists
    of paintings, flow between paintings, layout, and evaluation function.
    The solution was defined as a sequence of placement points for the paintings.

    \item  The second goal was to create a dataset for the painting placement problem.
    Four scenarios were defined – random, clustering, packing, and biased clustering.
    Multiple instances of the painting placement problem were generated for each scenario.

    \item  The thesis's third and central goal was to propose and implement a genetic approach for solving the painting placement problem.
    As described above, an individual is represented as multiple stochastic vectors in the novel genetic approach.
    Then, the crossover is implemented as vector addition, followed by normalization back to the stochastic vector.
    These vectors decode into one slicing tree, which recursively divides the space or wall where the paintings are placed.
    The second novel approach was to add a wildcard symbol $*$ to the possible values contained in the internal node of a slicing tree.
    Wildcard symbol $*$ can take up any value – $H$ for horizontal cut and $V$ for vertical cut.

    \item  The fourth goal was to evaluate the performance of the proposed genetic approach.
    It was achieved by creating a computational server to which a painting placement
    instance can be submitted and its painting placement solution obtained.
    Each instance in the dataset was submitted multiple times to the computation server to account for the statistical significance of the obtained results.
    These results were then presented and discussed.

    \item  The fifth and last goal was to discuss the results and suggest further improvements, extensions, and future work.
    Suggestions and result discussion were presented in chapter~\ref{ch:discussion}.
    One suggestion was creating an empty space inside the resulting layout by injecting dummy paintings into the slicing tree.
    This injection could be further used to work with non-rectangular layouts.
    Additionally, human operator assistance was mentioned, where the painting placement solution is presented to a human operator
    that further modifies it to his/her needs.

\end{enumerate}
