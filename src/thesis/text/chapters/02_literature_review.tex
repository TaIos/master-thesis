\chapter{Literature review}\label{ch:literature-review}

This chapter describes methods used in different fields to solve a similar problem to the painting placement problem.
It follows the methods mentioned in previous Chapter~\ref{ch:introduction} with
a more precise and in-depth explanation – facility layout problem (FLP) in section~\ref{sec:facility-layout-problem},
shelf-space planning problem in section~\ref{sec:shelf-space-planning} and sheet metal cutting problem in section~\ref{sec:sheet-metal-cutting}.

The similarity in all methods is that each place objects and evaluates the particular placement.
The main difference is their domain, which determines the objective function, i.e., the measure by which different solutions can be compared.
Also, there are differences between the given constraints, as some of them are more loose or strict,
depending on the intended use of the result.
A concise comparison, further explained in the following sections, follows.\\

\navesti{Facility layout problem} defines the flow between every facility pair and metric for measuring the distance between facilities.
The goal is to minimize the flow sum between all facilities.
Good results are mainly compactly placed facilities, where the ones with the highest flow between them are placed closer together.\\

\navesti{Shelf-space planning problem} has two main parts – dividing a shelf into rectangles and assigning a product to them.
This product placement on a shelf is evaluated using a profit function.
Similar to the sheet metal cutting problem, it differs from FLP
in not having any mutual relationship between placed products, i.e., there is no flow.
Another difference is that there are more products than available shelf space.
It implies that product choice must also be part of the shelf-space planning problem.
It is unique in using all shelf space in the first step.
However, empty space can still exist if the product has smaller dimensions than the shelf it is placed on.
Good results are mainly shelves that contain the largest amount of the most desired products at customer eye level.\\

\navesti{Sheet metal cutting problem} evaluates two aspects – compactness of the layout and distance taken by the path-cutting tool to cut all placed parts.
Similar to the shelf-space planning problem, placed parts have no mutual relationship.
However, unlike FLP and shelf-space planning, placed parts can have arbitrary shapes.
Good results depend on the balance between the two evaluated aspects.
If compactness is preferred, results will be more compactly packed, but the distance of the path-cutting tool might increase.
If, on the other hand, the lower cutting distance is preferred, it results in more common edges between the placed parts or clustering.\\

\section{Facility layout problem}\label{sec:facility-layout-problem}

The goal of the FLP is to place facilities, which are often represented as a rectangle,
in a given two-dimensional grid. The resulting placement tries to minimize
the cost function $c$.
Example used by~\cite{goncalvesBiasedRandomkeyGenetic2015} is shown is

\begin{equation}
    \argmin_{x \in K} c(x) = \sum\limits_{i=1}^{i=N}\sum\limits_{j=1}^{j=N}c_{i,j}\,f_{i,j}\,d_{i,j}
    \label{eq:flp-objective}
\end{equation}

where $K$ is the set of all possible facility placements, $f_{i,j} \in \mathbb{R^+}$ is the flow between facility $i$ and $j$, $c_{i,j} \in \mathbb{R^+}$
is price for a unit of distance between $i$, $j$ and $d_{i,j}$ is their distance.
Flow $f$ is defined to be symmetric, i.e., $f_{i,j} = f_{j,i}$~\cite{goncalvesBiasedRandomkeyGenetic2015, dunkerCoevolutionaryAlgorithmFacility2003}.
The FLP defined as such is NP-hard~\cite{liuMultiobjectiveParticleSwarm2018, goncalvesBiasedRandomkeyGenetic2015, friedrichIntegratedSlicingTree2018}.

Constraints to the facility placement is that no two facilities can overlap.
Some authors also consider facility orientation~\cite{liuMultiobjectiveParticleSwarm2018, tamHierarchicalApproachFacility1991}
to be part of the optimization problem.

Important part of the input to the FLP are the facility dimensions.
They are \textit{not} defined in terms of a pairs $w_i, h_i$, where $w_i$ would be width and $h_i$ height of a facility $i$.
Each facility is defined using its area $a_i$ and maximum aspect ratio $r_i \in \mathbb{R^+}$ which must satisfy equations~\ref{eq:flp-area} and~\ref{eq:flp-shape}.
\begin{equation}
    a_i = w_i h_i
    \label{eq:flp-area}
\end{equation}

\begin{equation}
    \dfrac{\argmax(w_i, h_i)}{\argmin(w_i, h_i)} < r_i
    \label{eq:flp-shape}
\end{equation}

Thus determining the width $w_i$ and height $h_i$ of the facility $i$ is part of the optimization problem.
Furthermore, there exists no FLP dataset that defines facilities in terms their width and height
– facility record in the dataset is always in the form of a $(a_i, r_i)$ pair~\cite{tamHierarchicalApproachFacility1991, dunkerCoevolutionaryAlgorithmFacility2003, liuSequencepairRepresentationMIPmodelbased2007}.

Metric used for measuring the distance in~\ref{eq:flp-objective} is important in practical applications.
Authors in~\cite{friedrichIntegratedSlicingTree2018} argue that using euclidean metric
to measure distance produces suboptimal results as, for example, material between facilities
hardly ever follows a direct route. Thus, they recommend using a countour-based metric.
In addition to solving the FLP, they also try to assign I/O points to the facilities
from which the distance is measured as opposed to measuring the distance from the
facility center.

\todo{popsat metody reseni, hlavne genetiku a slicing trees}









\section{Shelf-space planning}\label{sec:shelf-space-planning}

\newpage
\section{Sheet metal cutting}\label{sec:sheet-metal-cutting}

The sheet metal cutting problem evaluates two aspects – compactness of the layout, also called pattern efficiency, and distance taken
by the path-cutting tool to cut all placed parts.
Also, placed parts that determine the pattern efficiency can be irregular, for example, forming a polygon~\cite{hajadLaserCuttingPath2019}.

Authors in~\cite{vijayanandHeuristicGeneticApproach2015}
use both of these aspects to formulate an objective function $F$ that defines a sheet metal cutting problem as

\begin{equation}
    \min F = w_1\,f_1 + w_2\,f_{2}\,,
    \label{eq:metal-sheet-cutting}
\end{equation}

where $f_1$ is pattern efficiency, which is calculated as $\dfrac{\sum\limits_{i=1}^{N}a_i}{WH}$,
with $N$ being number of placed parts, $a_i$ the area of the $i$-th placed part, $W$ width and $H$ height of the
the smallest rectangle that can contain all placed parts, and $f_2$ the distance
needed by the path-cutting tool to cut all the placed parts.
Weights $w_1 \in \realpos, w_2 \in \realpos$ balance the bias towards pattern efficiency or path-cutting tool distance.

Some variants of the sheet metal cutting problem neglect the pattern efficiency and try only to find the shortest
path of a given placed parts~\cite{kandasamyEffectiveLocationMicro2020}.
Then, the sheet metal cutting problem can be reformulated as the GTSP (Generalized Traveling Salesman Problem)~\cite{hajadLaserCuttingPath2019}.
The sheet metal cutting problem is thus considered an NP-hard problem~\cite{vijayanandHeuristicGeneticApproach2015}.

Authors in~\cite{vijayanandHeuristicGeneticApproach2015} solve sheet metal cutting problem using a genetic algorithm
where an individual is represented as a 3D chromosome.
It contains the cluster size written as a binary number, the sequence of placed parts, and their orientation.
This 3D chromosome is then input to the placing heuristic, which places the parts.

The solution proposed in~\cite{kandasamyEffectiveLocationMicro2020} considers only path-cutting tool distance.
However, they partition each placed part into multiple segments using micro joints, i.e., points at the edges of the part.
The path-cutting tool then cuts these segments instead of the whole part at once.
They propose a genetic algorithm with the 2D chromosome – containing angles that determine the placement of the micro joints and cutting sequence of these segments.
The chromosome is then directly used by the path-cutting tool that starts cutting segments according to the cutting sequence defined in the second part of the chromosome.



