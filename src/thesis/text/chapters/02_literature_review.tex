\chapter{Literature review}\label{ch:literature-review}

This chapter describes methods used in different fields to solve a similar problem to the painting placement problem.
It follows the methods mentioned in previous Chapter~\ref{ch:introduction} with
a more precise and in-depth explanation – facility layout problem (FLP) in section~\ref{sec:facility-layout-problem},
shelf-space planning problem in section~\ref{sec:shelf-space-planning} and sheet metal cutting problem in section~\ref{sec:sheet-metal-cutting}.

The main difference in all methods is their domain, which determines the objective function, i.e.,
the measure by which different solutions can be compared.
Also, there are differences between the given constraints, as some of them are more loose or strict,
depending on the intended use of the result.
A concise comparison, further explained in the following sections, follows.\\

\navesti{Facility layout problem} defines the \textit{flow} between every facility pair and metric for measuring the distance between facilities.
The goal is to minimize the \textit{flow} sum between all facilities.
Good results are mainly compactly placed facilities, where the ones having the
highest flow between them are placed closer together with respect to the metric.\\

\navesti{Shelf-space planning problem} has two main parts – dividing a shelf into rectangles and assigning a product to them.
This product placement into a shelf is evaluated using a \textit{profit function}. Similar to sheet metal cutting problem, it differs from FLP
in not having any mutual relationship between placed products, i.e., there is no \textit{flow}.
Another difference is that there are more products than available shelf-space.
This implies that product choice also has to be part of shelf-space planning problem.
It is unique in using all shelf-space in the first step.
However, empty space can still exist if the product has smaller dimensions than the shelf it is placed on.
Good results are mainly shelves that contain the largest amount of the most desired products at customer eye level.\\

\navesti{Sheet metal cutting problem} evaluates two aspects – compactness of the layout and distance taken by the path-cutting tool to cut all placed parts.
Similar to shelf-space planning problem, there is no mutual relationship between placed parts.
Good results depend on the balance between the two evaluated aspects.
If compactness is preferred, results will be more compactly packed, but the distance of the path-cutting tool might increase.
If, on the other hand, the lower cutting distance is preferred, it results in more common edges between the placed parts or clustering.\\


\section{Facility layout problem}\label{sec:facility-layout-problem}

The goal of the FLP (Facility Layout Problem) is to place facilities, which are often represented as a rectangle,
in a given two-dimensional grid that is also rectangular.
In addition, a flow exists between each pair of facilities, i.e., the number that defines whether it is advantageous to place facilities close together.

Authors in~\cite{goncalvesBiasedRandomkeyGenetic2015} define facility layout problem using a cost function $c$ as

\begin{equation}
    \argmin_{x \in K} c(x) = \sum\limits_{i=1}^{i=N}\sum\limits_{j=1}^{j=N}c_{i,j}\,f_{i,j}\,d_{i,j}\,,
    \label{eq:flp-objective}
\end{equation}

where $K$ is the set of all possible facility placements, $N$ is the number of facilities, $f_{i,j} \in \mathbb{R^+}$ is the flow between facility $i$ and $j$, $c_{i,j} \in \mathbb{R^+}$
is price for a unit of distance between $i$, $j$ and $d_{i,j}$ is their distance.
Flow $f$ is defined to be symmetric, i.e., $f_{i,j} = f_{j,i}$.
The constraint to the facility placement problem is that no facilities can overlap.
The FLP defined as such is NP-hard~\cite{liuMultiobjectiveParticleSwarm2018, goncalvesBiasedRandomkeyGenetic2015, friedrichIntegratedSlicingTree2018}.

An important part of the input to the FLP is the facility dimensions.
They are not defined as $w_i, h_i$ pairs, where $w_i$ is the width and $h_i$ height of a facility $i$.
Each facility is defined using its area $a_i$ and maximum aspect ratio $r_i \in \mathbb{R^+}$ which must satisfy equations~\ref{eq:flp-area} and~\ref{eq:flp-shape}.
\begin{equation}
    a_i = w_i h_i
    \label{eq:flp-area}
\end{equation}

\begin{equation}
    \dfrac{\argmax(w_i, h_i)}{\argmin(w_i, h_i)} < r_i
    \label{eq:flp-shape}
\end{equation}

Thus, determining the width $w_i$ and height $h_i$ of the facility $i$ is part of the facility layout problem.
Furthermore, no FLP dataset defines facilities in terms of their width and height.
Facility records in the datasets are always in the form of a $(a_i, r_i)$ pair~\cite{tamHierarchicalApproachFacility1991, dunkerCoevolutionaryAlgorithmFacility2003, liuSequencepairRepresentationMIPmodelbased2007}.

Metric $d$ in equation~\ref{sec:facility-layout-problem} used for measuring the distance between facilities is important in practical applications of the facility layout problem.
Authors in~\cite{friedrichIntegratedSlicingTree2018} argue that using Euclidean $L2$ norm
to measure facility distance produces suboptimal results as, for example, transportation of material between facilities
hardly ever follows a direct route.
Thus, they recommend using a contour-based metric instead.
In addition, they also try to assign I/O points to the facilities, which are points from which the distance $d$ is measured.
Some authors also consider facility orientation~\cite{liuMultiobjectiveParticleSwarm2018, tamHierarchicalApproachFacility1991}
to be part of the facility layout problem.

Solution methods differ in the exact definition of the FLP problem.
Authors in~\cite{liuMultiobjectiveParticleSwarm2018} solve the UA-FLP (Unequal Area FLP), the variant of FLP where the placed facilities have unequal areas.
They propose a solution using particle swarm optimization.
Authors in~\cite{changSlicingTreeRepresentation2013} proposed a solution to UA-FLP combining harmony search and slicing tree.
They represent harmony vector as coding of a slicing tree, which has two parts – the first part being a binary string determining the slicing tree node type, i.e., inner or leaf, and the second part codes the content of the slicing tree leaves.
Authors in~\cite{karyantamGeneticAlgorithmsFunction1992} use a similar genetic approach with a chromosome defined as a post-order traversal of a slicing tree.
A genetic solution for the UA-FLP proposed in~\cite{goncalvesBiasedRandomkeyGenetic2015} uses a BRKGA (Biased Random Key Genetic Algorithm), where the chromosome contains facility sequence random keys,
aspect ratios and position of the first facility.
Next, an iterative greedy heuristic with the above chromosome as an input is used.
Authors in~\cite{friedrichIntegratedSlicingTree2018} propose a unique solution to the FLP called parallel tempting based on simulated annealing.

\section{Shelf-space planning}\label{sec:shelf-space-planning}

\newpage
\section{Sheet metal cutting}\label{sec:sheet-metal-cutting}

The sheet metal cutting problem evaluates two aspects – compactness of the layout, also called pattern efficiency, and distance taken
by the path-cutting tool to cut all placed parts.
Also, placed parts that determine the pattern efficiency can be irregular, for example, forming a polygon~\cite{hajadLaserCuttingPath2019}.

Authors in~\cite{vijayanandHeuristicGeneticApproach2015}
use both of these aspects to formulate an objective function $F$ that defines a sheet metal cutting problem as

\begin{equation}
    \min F = w_1\,f_1 + w_2\,f_{2}\,,
    \label{eq:metal-sheet-cutting}
\end{equation}

where $f_1$ is pattern efficiency, which is calculated as $\dfrac{\sum\limits_{i=1}^{N}a_i}{WH}$,
with $N$ being number of placed parts, $a_i$ the area of the $i$-th placed part, $W$ width and $H$ height of the
the smallest rectangle that can contain all placed parts, and $f_2$ the distance
needed by the path-cutting tool to cut all the placed parts.
Weights $w_1 \in \realpos, w_2 \in \realpos$ balance the bias towards pattern efficiency or path-cutting tool distance.

Some variants of the sheet metal cutting problem neglect the pattern efficiency and try to only find the shortest
path of a given placed parts~\cite{kandasamyEffectiveLocationMicro2020}.
Then, the sheet metal cutting problem can be reformulated as the Generalized Traveling Salesman Problem (GTSP)~\cite{hajadLaserCuttingPath2019}.
The sheet metal cutting problem is thus considered to be NP-hard problem~\cite{vijayanandHeuristicGeneticApproach2015}.

Authors in~\cite{vijayanandHeuristicGeneticApproach2015} solve sheet metal cutting problem using a genetic algorithm
where an individual is represented as a 3D chromosome.
It contains the cluster size written as a binary number, the sequence of placed parts, and their orientation.
This 3D chromosome is then input to the placing heuristic, which places the parts.

The solution proposed in~\cite{kandasamyEffectiveLocationMicro2020} considers only path-cutting tool distance.
However, they partition each placed part into multiple segments using micro joints, i.e., points at the edges of the part.
The path-cutting tool then proceeds by cutting these segments instead of the whole part at once.
They propose a genetic algorithm with the 2D chromosome – containing angles that determine the placement of the micro joints and cutting sequence of these segments.
The chromosome is then directly used by the path-cutting tool that starts cutting segments according to the cutting sequence defined in the second part of the chromosome.


