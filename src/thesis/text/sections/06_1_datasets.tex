\section{Dataset}\label{sec:dataset}

As mentioned in Chapter~\ref{ch:literature-review}, there are no datasets in the
literature that would satisfy the definition of painting placement instance in~\ref{eq:painting-placement-instance}.
Thus, all datasets are exclusively created by the author and can be used by other researchers
for benchmarking their solutions. Generation is performed using a Python programming language in combination with Jupyter Notebook.
Both the datasets and Python code can be found in the attached medium.

There are 4 scenarios that are tested – random, cluster, packing and biased clustering.
Also, there is one dataset that describes the painting placement at the London National Gallery in figure~\ref{fig:london-wall}. \\

\navesti{Random} scenario contains randomly generated datasets that are mainly used for performance testing.\\

\navesti{Cluster} scenario tests the ability to form clusters.
This is achieved as dividing the paintings to groups. Paintings belonging to the
same group have increased flow between them. Paintings from the distinct group have flow between
them set to 0. \\

\navesti{Packing} scenario is the same as the \textit{random} with the only difference that the layout area
is equal to the area of all paintings summed together. This tests the ability to create compact solutions.\\

\navesti{Biased clustering} scenario is similar to the \textit{cluster} with the only difference
that the evaluation function is set in a way that the formation of clusters is advantageous in certain parts
of the layout. This test the ability to not only create clusters but also to force their formation in certain parts of the layout.\\

Table~\ref{tab:dataset-params} describes the generation parameters of each dataset. Value \textit{layout area ratio}
describes the ratio between the area of the layout and the painting area sum. It can thus be computed as

\[
    \dfrac{\sum\limits_{i=1}^{N} w_i h_i}{W \times H}\,,
\]

where $w_i$ is width, $h_i$ is height of painting $i$, $W$ is width, and $H$ is height of the layout.
If the \textit{layout area ratio} is set to $1$, it means a preference for more compact layouts. On the other hand,
increasing this value implies the presence of free space in the resulting layout.

The value \textit{max painting ratio} controls the maximum aspect ratio between width $w$ and height $h$ of each painting.
It is computed as

\[
    \dfrac{\max(w,h)}{\min(w,h)}\,.
\]

Increasing \textit{max painting ratio} implies the possibility for generation of paintings
that are very thin, i.e. $w \ll h$ or $h \ll w$. On the other hand, setting the value to 1
implies that every generated painting is square.

\begin{table}[]
    \begin{tabular}{|c|c|c|c|c|c|c|c|}
        \hline
        &
        \begin{tabular}[c]{@{}c@{}}
            layout\\ area ratio
        \end{tabular} &
        \begin{tabular}[c]{@{}c@{}}
            max painting\\ width
        \end{tabular} &
        \begin{tabular}[c]{@{}c@{}}
            max painting\\ height
        \end{tabular} &
        \begin{tabular}[c]{@{}c@{}}
            max painting\\ ratio
        \end{tabular} &
        \begin{tabular}[c]{@{}c@{}}
            flow\\ min
        \end{tabular} &
        \begin{tabular}[c]{@{}c@{}}
            flow\\ max
        \end{tabular} &
        \begin{tabular}[c]{@{}c@{}}
            eval\\ function
        \end{tabular} \\ \hline
        random        & 1.2 & 10 & 10 & 3 & 0 & 4 & $f(x,y) = 0$ \\ \hline
        cluster       & 1.2 & 10 & 10 & 3 & - & - & $f(x,y) = 0$ \\ \hline
        packing       & 1   & 10 & 10 & 3 & 0 & 4 & $f(x,y) = 0$ \\ \hline
        \begin{tabular}[c]{@{}c@{}}
            biased\\ clustering
        \end{tabular} & 2   & 10 & 10 & 3 & - & - & -            \\ \hline
    \end{tabular}
    \caption{Parameters used to generate testing scenarios. Left-out values marked with - are discussed later in the text.}
    \label{tab:dataset-params}
\end{table}

