\section{Facility layout problem}\label{sec:facility-layout-problem}

The goal of the FLP is to place facilities, which are often represented as a rectangle,
in a given two-dimensional grid that is also rectangular.
In addition, a flow exists between each pair of facilities, i.e., the number that defines whether it is advantageous to place facilities close together.

Authors in~\cite{goncalvesBiasedRandomkeyGenetic2015} define facility layout problem using a cost function $c$ as

\begin{equation}
    \argmin_{x \in K} c(x) = \sum\limits_{i=1}^{i=N}\sum\limits_{j=1}^{j=N}c_{i,j}\,f_{i,j}\,d_{i,j}\,,
    \label{eq:flp-objective}
\end{equation}

where $K$ is the set of all possible facility placements, $N$ is the number of facilities, $f_{i,j} \in \mathbb{R^+}$ is the flow between facility $i$ and $j$, $c_{i,j} \in \mathbb{R^+}$
is price for a unit of distance between $i$, $j$ and $d_{i,j}$ is their distance.
Flow $f$ is defined to be symmetric, i.e., $f_{i,j} = f_{j,i}$.
The constraint to the facility placement problem is that no facilities can overlap.
The FLP defined as such is NP-hard~\cite{liuMultiobjectiveParticleSwarm2018, goncalvesBiasedRandomkeyGenetic2015, friedrichIntegratedSlicingTree2018}.

An important part of the input to the FLP is the facility dimensions.
They are not defined $w_i, h_i$ pairs, where $w_i$ is the width and $h_i$ height of a facility $i$.
Each facility is defined using its area $a_i$ and maximum aspect ratio $r_i \in \mathbb{R^+}$ which must satisfy equations~\ref{eq:flp-area} and~\ref{eq:flp-shape}.
\begin{equation}
    a_i = w_i h_i
    \label{eq:flp-area}
\end{equation}

\begin{equation}
    \dfrac{\argmax(w_i, h_i)}{\argmin(w_i, h_i)} < r_i
    \label{eq:flp-shape}
\end{equation}

Thus, determining the width $w_i$ and height $h_i$ of the facility $i$ is part of the facility layout problem.
Furthermore, no FLP dataset defines facilities in terms of their width and height
– facility records in the datasets are always in the form of a $(a_i, r_i)$ pair~\cite{tamHierarchicalApproachFacility1991, dunkerCoevolutionaryAlgorithmFacility2003, liuSequencepairRepresentationMIPmodelbased2007}.

Metric $d$ in equation~\ref{sec:facility-layout-problem} used for measuring the distance between facilities is important in practical applications of the facility layout problem.
Authors in~\cite{friedrichIntegratedSlicingTree2018} argue that using Euclidean $L2$ norm
to measure facility distance produces suboptimal results as, for example, transportation of material between facilities
hardly ever follows a direct route.
Thus, they recommend using a contour-based metric instead.
In addition, they also try to assign I/O points to the facilities from which the distance is measured as opposed to measuring the distance from the facility center.
Some authors also consider facility orientation~\cite{liuMultiobjectiveParticleSwarm2018, tamHierarchicalApproachFacility1991}
to be part of the facility layout problem.

Solution methods differ in the exact definition of the FLP problem.
Authors in~\cite{liuMultiobjectiveParticleSwarm2018} solve the UA-FLP, the variant of FLP where the placed facilities have unequal areas.
They propose a solution using particle swarm optimization.
Authors in~\cite{changSlicingTreeRepresentation2013} proposed a solution to UA-FLP combining harmony search and slicing tree.
They represent harmony vector as coding of a slicing tree, which has two parts – the first part being a binary string determining the slicing tree node type, i.e., inner or leaf, and the second part codes the content of the slicing tree leaves.
Authors in~\cite{karyantamGeneticAlgorithmsFunction1992} use a similar genetic approach with a chromosome defined as a post-order traversal of a slicing tree.
A genetic solution for the UA-FLP proposed in~\cite{goncalvesBiasedRandomkeyGenetic2015} uses a BRKGA, where the chromosome contains facility sequence random keys,
aspect ratios and position of the first facility.
Next, an iterative greedy heuristic with the above chromosome as an input is used.
Authors in~\cite{friedrichIntegratedSlicingTree2018} propose a unique solution to the FLP called parallel tempting based on simulated annealing.
