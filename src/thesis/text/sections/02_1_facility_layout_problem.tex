\section{Facility layout problem}\label{sec:facility-layout-problem}

The goal of the FLP is to place facilities, which are often represented as a rectangle,
in a given two-dimensional grid. The resulting placement tries to minimize
the cost function $c$.
Example used by~\cite{goncalvesBiasedRandomkeyGenetic2015} is shown is

\begin{equation}
    \argmin_{x \in K} c(x) = \sum\limits_{i=1}^{i=N}\sum\limits_{j=1}^{j=N}c_{i,j}\,f_{i,j}\,d_{i,j}
    \label{eq:flp-objective}
\end{equation}

where $K$ is the set of all possible facility placements, $f_{i,j} \in \mathbb{R^+}$ is the flow between facility $i$ and $j$, $c_{i,j} \in \mathbb{R^+}$
is price for a unit of distance between $i$, $j$ and $d_{i,j}$ is their distance.
Flow $f$ is defined to be symmetric, i.e., $f_{i,j} = f_{j,i}$~\cite{goncalvesBiasedRandomkeyGenetic2015, dunkerCoevolutionaryAlgorithmFacility2003}.
The FLP defined as such is NP-hard~\cite{liuMultiobjectiveParticleSwarm2018, goncalvesBiasedRandomkeyGenetic2015, friedrichIntegratedSlicingTree2018}.

Constraints to the facility placement is that no two facilities can overlap.
Some authors also consider facility orientation~\cite{liuMultiobjectiveParticleSwarm2018, tamHierarchicalApproachFacility1991}
to be part of the optimization problem.

Important part of the input to the FLP are the facility dimensions.
They are \textit{not} defined in terms of a pairs $w_i, h_i$, where $w_i$ would be width and $h_i$ height of a facility $i$.
Each facility is defined using its area $a_i$ and maximum aspect ratio $r_i \in \mathbb{R^+}$ which must satisfy equations~\ref{eq:flp-area} and~\ref{eq:flp-shape}.
\begin{equation}
    a_i = w_i h_i
    \label{eq:flp-area}
\end{equation}

\begin{equation}
    \dfrac{\argmax(w_i, h_i)}{\argmin(w_i, h_i)} < r_i
    \label{eq:flp-shape}
\end{equation}

Thus determining the width $w_i$ and height $h_i$ of the facility $i$ is part of the optimization problem.
Furthermore, there exists no FLP dataset that defines facilities in terms their width and height
– facility record in the dataset is always in the form of a $(a_i, r_i)$ pair~\cite{tamHierarchicalApproachFacility1991, dunkerCoevolutionaryAlgorithmFacility2003, liuSequencepairRepresentationMIPmodelbased2007}.

Metric used for measuring the distance in~\ref{eq:flp-objective} is important in practical applications.
Authors in~\cite{friedrichIntegratedSlicingTree2018} argue that using euclidean metric
to measure distance produces suboptimal results as, for example, material between facilities
hardly ever follows a direct route. Thus, they recommend using a countour-based metric.
In addition to solving the FLP, they also try to assign I/O points to the facilities
from which the distance is measured as opposed to measuring the distance from the
facility center.

\todo{popsat metody reseni, hlavne genetiku a slicing trees}








