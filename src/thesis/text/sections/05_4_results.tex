\clearpage
\newpage


\section{Results}\label{sec:results}
This section presents a painting placement solution to the painting placement instances in table~\ref{tab:instances}.
Obtained solutions are discussed in the following subsections –
random scenario instances in~\ref{subsec:random-scenario},
packing scenario instances in~\ref{subsec:packing-scenario},
clustering scenario instances in~\ref{subsec:clustering-scenario},
and London National Gallery instance~\ref{subsec:london-gallery-wall}.
Hyperparameter values used to obtain results in this section are in table~\ref{tab:hyperparameters-values}.
Statistics for the last iteration of the genetic algorithm for each instance are in table~\ref{tab:statistics}.

Hyperparameter values in table~\ref{tab:hyperparameters-values} are set to their recommended values
from hyperparameter testing in section~\ref{sec:hyper-parameters}.
The hyperparameter not set to the recommended values is \verb|maxNumberOfIter|.
It is set to 500 instead of a value from the recommended range \numrange{100}{150}.
The reason is to possibly find a better painting placement solution in exchange for more computation time.

The recommendation to remove orientation penalization by setting \verb|orientationWeights| to $\langle 1,1,1\rangle$ is followed.
As described in hyperparameter testing, it should produce a population with a better on-average objective value and a faster-decreasing trend in objective value.
Also, the population size is set to 75 times the instance size.
It is the midpoint of the recommended interval \numrange{50}{100}.
Lastly, population division is set to the same values as in listing~\ref{lst:computation-submission-dataset}.
It means keeping an elitism strategy and injecting random individuals.


\begin{table}[h!]
    \caption{Hyperparameter values used to obtain results}
    \label{tab:hyperparameters-values}
    \begin{threeparttable}
        \begin{tabular}{ll}
            \hline
            \textbf{Hyperparameter}                           & \textbf{Value}             \\ \hline
            \verb|maxNumberOfIter|                            & 500                        \\ \hline
            \verb|populationSize|                             & 75 times the instance size \\ \hline
            \verb|maximumWildCardCount|                       & 1                          \\ \hline
            \verb|orientationWeights|                         & $\langle 1,1,1 \rangle$    \\ \hline
            \verb|populationDivisionCounts|                   & elitism, random            \\ \hline
            \verb|initialPopulationDivisionCounts|            & 0.7 random, 0.3 greedy     \\ \hline
            \verb|overlappingPenalizationConstant| & \begin{tabular}[c]{@{}l@{}}
                                                         4 times the diagonal length\\ of the layout
            \end{tabular} \\ \hline
            \verb|outsideOfAllocatedAreaPenalizationConstant| & 0                          \\ \hline
        \end{tabular}
        \begin{tablenotes}
            \small
            \item Hyperparameter description is in table~\ref{tab:hyperparameters-description}.
        \end{tablenotes}
    \end{threeparttable}
\end{table}

\begin{table}[h!]
    \caption{Statistics of the last iteration}
    \label{tab:statistics}
    \begin{threeparttable}
        \begin{tabular}{lllll}
            \hline
            \textbf{Instance name} &
            \textbf{\begin{tabular}[c]{@{}l@{}}
                        Best obj.\\ value
            \end{tabular}} &
            \textbf{\begin{tabular}[c]{@{}l@{}}
                        Worst obj.\\ value
            \end{tabular}} &
            \textbf{\begin{tabular}[c]{@{}l@{}}
                        Obj.\\ mean
            \end{tabular}} &
            \textbf{\begin{tabular}[c]{@{}l@{}}
                        Standard\\ deviation
            \end{tabular}} \\ \hline
            random\_10            & 1136.11 & 3438.01  & 1640.91 & 460.76  \\ \hline
            random\_20            & 5417.8  & 11629.59 & 7209.22 & 1400.13 \\ \hline
            packing\_10           & 669.48  & 2369.11  & 1142.96 & 383.22  \\ \hline
            packing\_20           & 6219.13 & 12283.41 & 8140.29 & 1450.09 \\ \hline
            cluster\_3\_6         & 3921.68 & 11121.96 & 6317.36 & 1844.99 \\ \hline
            cluster\_4\_5         & 4887.04 & 12088.99 & 7177.9  & 1767.67 \\ \hline
            london\_gallery\_wall & 2759.65 & 15764.31 & 5440.9  & 2633.81 \\ \hline
        \end{tabular}
        \begin{tablenotes}
            \small
            \item Instance description is in table~\ref{tab:instances}.
        \end{tablenotes}
    \end{threeparttable}
\end{table}

\newpage

\subsection{Random scenario}\label{subsec:random-scenario}

Visualization of the painting placement solution to the random\_10 instance
is in figure~\ref{fig:results:visualization-random-10}
and for random\_20 instance in figure~\ref{fig:results:visualization-random-20}.

For the random\_10 instance, we can see that there are no overlappings, and only painting~6 is partially outside the layout.
Also, due to the eval function set to $f(x,y) = x+y$, smaller paintings
are placed towards the bottom left corner.
This painting placement solution can be considered successful,
because it adheres to all penalizations that are applied.

For the random\_20 instance, we can see that there are four overlappings.
Similarly to the random\_10 instance, the eval function forced the placement of the
smaller paintings to the bottom left corner, and one painting is placed outside the layout.
Due to the four overlappings, it is only a partially successful painting placement solution.
On the other hand, the solution adheres to all other penalizations.
It might suggest further increasing the overlapping penalization constant~$\lambda$.

\subsection{Packing scenario}\label{subsec:packing-scenario}
Visualization of the painting placement solution to the packing\_10 instance
is in figure~\ref{fig:results:visualization-packing-10}
and for packing\_20 instance in figure~\ref{fig:results:visualization-packing-20}.

For the packing\_10 instance, we can see no overlappings, and paintings~2,3, and 4 are partially outside the layout.
It is considered a success, as the packing scenario tests the ability to form compact solutions by setting the instance generation parameter
layout area ratio to one, which means that the area of the layout equals the area of all paintings summed together.
However, an improvement in compactness can still be gained by moving paintings 1 and 2 downwards.
It could be implemented using a post-optimization, suggested in subsection~\ref{subsec:post-optimization}.
The eval function is absent in the packing scenario, so there is no preference for any part of the layout.

For the packing\_20 instance, we can see seven overlapping pairs.
It is not considered a success as there is no eval function present and overlapping penalization constant~$\lambda$
should be the main source of improvement if the objective function~\ref{eq:objective}.
However, the packing complexity for 20 paintings is high, and it might be impossible to fit all paintings to the layout without overlapping or being outside of the allocated area.
Nevertheless, the suggestion is to increase the overlapping penalization constant~$\lambda$.


\subsection{Clustering scenario}\label{subsec:clustering-scenario}

Two clustering instances are tested, with visualization for cluster\_3\_6 instance
in figure~\ref{fig:results:visualization-cluster-3-6} and for cluster\_4\_5 instance in figure~\ref{fig:results:visualization-cluster-4-5}.
The clustering scenario tested the ability to place paintings inside clusters.
Clusters are defined using a flow between paintings, which is set to 0 between paintings from different
clusters and nonzero for the paintings from the same clusters.
Visualization of the flow for cluster\_3\_6 instance is in the appendix on the right of the figure~\ref{fig:instance-flow}.

For the cluster\_3\_6 instance, we can see the successful formation of clusters.
However, seven overlapping pairs exist between paintings in the same cluster.
It might suggest that the flow is too high or the overlapping penalization constant $\lambda$ is too low.


For the cluster\_4\_5 instance, we can see a partially successful formation of clusters.
It might be caused by the increased complexity as there are four clusters instead of three for the cluster\_3\_6 instance.
We can see overlappings between the paintings from the same cluster and even between clusters.
The same suggestion as for the cluster\_3\_6 instance might be given – the flow is set too high or that the overlapping penalization constant $\lambda$ is set too low.

\subsection{London Gallery Wall}\label{subsec:london-gallery-wall}

The last tested instance is london\_gallery\_wall.
It represents the painting placement at the London National Gallery from figure~\ref{fig:london-wall}.

Visualization of the result is in figure~\ref{fig:results:visualization-london-gallery-wall}.
We can see no overlappings, and two paintings are placed outside the wall.

The proposed solution can reconstruct the interactions between paintings using the flow.
It means that the paintings that are placed close together in the London National Gallery from figure~\ref{fig:london-wall}
are also placed together in the visualization of the painting placement solution in figure~\ref{fig:results:visualization-london-gallery-wall}.
Concrete flow values that are used are in the attached medium.


\clearpage
\newpage

\begin{figure}[h!]
    \includegraphics[width=0.8\textwidth, center]{visualizations/visualization_random_10}
    \caption[Painting placement solution for the random\_10 instance]
        {Painting placement solution for the random\_10 instance.
    There are no overlappings.}
    \label{fig:results:visualization-random-10}
\end{figure}

\begin{figure}[h!]
    \includegraphics[width=0.8\textwidth, center]{visualizations/visualization_random_20}
    \caption[Painting placement solution for the random\_20 instance]
        {Painting placement solution for the random\_20 instance.
    Four overlapping pairs exist (15–18, 15–16, 3–10, 9–11).}
    \label{fig:results:visualization-random-20}
\end{figure}

\begin{figure}[h!]
    \includegraphics[width=0.8\textwidth, center]{visualizations/visualization_packing_10}
    \caption[Painting placement solution for the packing\_10 instance]
        {Painting placement solution for the packing\_10 instance.
    There are no overlappings.}
    \label{fig:results:visualization-packing-10}
\end{figure}

\begin{figure}[h!]
    \includegraphics[width=0.8\textwidth, center]{visualizations/visualization_packing_20}
    \caption[Painting placement solution for the packing\_20 instance]
        {Painting placement solution for the packing\_20 instance.
    There are seven overlapping pairs (17–18, 1–6, 1–11, 11–13, 10–19, 8–19, 7–8).}
    \label{fig:results:visualization-packing-20}
\end{figure}

\begin{figure}[h!]
    \includegraphics[width=0.8\textwidth, center]{visualizations/visualization_cluster_3_6}
    \caption[Painting placement solution for the cluster\_3\_6 instance]
        { Painting placement solution for the cluster\_3\_6 instance.
    Three groups of paintings, \numrange{1}{6}, \numrange{7}{12}, and \numrange{13}{18}, are marked using different colors.
    There are seven overlapping pairs.}
    \label{fig:results:visualization-cluster-3-6}
\end{figure}

\begin{figure}[h!]
    \includegraphics[width=0.8\textwidth, center]{visualizations/visualization_cluster_4_5}
    \caption[Painting placement solution for the cluster\_4\_5 instance]
        { Painting placement solution for the cluster\_4\_5 instance.
    Four groups of paintings, \numrange{1}{5}, \numrange{6}{10}, \numrange{11}{15}, and \numrange{15}{20}, are marked using different colors.
    There are seven overlapping pairs.}
    \label{fig:results:visualization-cluster-4-5}
\end{figure}

\begin{figure}[h!]
    \includegraphics[width=0.8\textwidth, center]{visualizations/visualization_london_gallery_wall}
    \caption[Painting placement solution for the london\_gallery\_wall instance]
        { Painting placement solution for the london\_gallery\_wall instance.
    Paintings are marked using their original position in figure~\ref{fig:london-wall}.
    There are no overlappings.}
    \label{fig:results:visualization-london-gallery-wall}
\end{figure}
