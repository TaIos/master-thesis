\section{Genetics}\label{sec:genetics}

This section describes important genetic terms that are used throughout the thesis.
They are allele, gene, chromosome, individual, population, crossover, mutation, and reproductive plan.
Also, in subsection~\ref{subsec:schema-theorem}, the integral part of genetics called the Schema Theorem is described.

First, it is essential to describe what the genetic approach means.
The genetic approach was first introduced by Holland in 1975
to solve optimization problems where it is computationally infeasible to
find an optimal solution by enumerating all possible solutions~\cite{hollandAdaptationNaturalArtificial1975}.

This genetic approach is inspired by nature and Darwin’s Theory of Evolution
– a population of individuals evolving over time.
Individuals more adapted to the environment are
more likely to survive and thus pass their genes to the next generation.
Thus, over time, the population should converge to the state where the adaptation to the environment is the highest~\cite{darwinOriginSpeciesMeans2009}.

Holland in~\cite{hollandAdaptationNaturalArtificial1975} defines several structures that reassemble this natural process.
The most important ones are described in the rest of this section.\\

\navesti{Allele} represents a concrete value that a gene can have.
It can be thus described as a set of alternatives to choose from. \\

\navesti{Gene} is a structure composed of alleles.
It often describes one trait or characteristic. \\

\navesti{Chromosome} is a structure composed of genes.
Thus it is an amalgam of characteristics described by genes.\\

\navesti{Individual} is defined by its chromosome and represents a solution to the problem or a structure from which a solution can be constructed.
A numeric value called \definice{fitness} can be assigned to each individual, representing how well the individual performs in an environment. \\

\navesti{Population} is a set of individuals.
It can be interpreted as a subset of possible solutions.\\

One concrete example of the above-mentioned definitions is in figure~\ref{fig:population}.
There are two individuals in the population, with their chromosomes having two genes.
The first gene is a vector containing permutation with alleles of 1 to 5.
The second gene is a string vector
with alleles having values $H$ or $V$ (cut types from subsec.~\ref{subsec:individual-decoding}).
Lastly, each individual has a fitness value.
Thus, because A’s fitness $30.5$ is greater than B’s fitness $9.8$, we can say that individual A performs better than B.


\begin{figure}
    \includegraphics[width=1.1\textwidth, left]{population}
    \caption[Population example]{Example of a population composed of two individuals.}
    \label{fig:population}
\end{figure}

For the structures defined above, multiple operations called \definice{genetic operators} or simply operators
are defined by Holland and used by other researchers following his work.
Genetic operators aim to create new individual/s using individuals already present in a population as input.
Two of them that are used in this thesis are described below.\\


\navesti{Crossover} genetic operator takes two individuals as input and, by recombination
of their alleles in their genes, produces a new individual/s called \definice{offspring} or \definice{child}.\\

\navesti{Mutation} genetic operator takes one individual as input and produces a new one
which may have some of its alleles replaced by different ones at random.\\

Additionally, there needs to be a process that transforms a population
to a new one.
This process is called the reproductive plan.
Also, there is a special term for the population to which the reproductive plan is applied.
\\

\navesti{Reproductive plan} is a process that takes a population on input and produces a modified population on the output
by using mainly genetic operators.\\

\navesti{Initial population} is the population before the first application of the reproductive plan.\\

\navesti{Generation} $k$ is the population after applying the reproductive plan $k$-times to it.\\

\newpage

Finally, a genetic algorithm uses all the processes mentioned above to
find a solution to some problem.\\

\navesti{Genetic algorithm} or \definice{genetic approach} applies a reproductive plan
on an initial population until the stopping condition is met with the
goal of finding a (sub)-optimal solution to the problem.\\

One example of a genetic approach is in figure~\ref{fig:reproductive-plan}.
At the start, an initial population of individuals is generated.
Then, the reproductive plan is applied until the stopping condition is met.
Application of the reproductive plan creates the next generation by using crossover and mutation genetic operators.

\begin{figure}[h]
    \includegraphics[width=0.65\textwidth, center]{reproductive_plan}
    \caption[Example of a genetic approach]{
        Example of a genetic approach.
        It uses a reproductive plan that creates the next generation by applying crossover and mutation.}
    \label{fig:reproductive-plan}
\end{figure}

\subsection{Schema Theorem}\label{subsec:schema-theorem}

Holland in~\cite{hollandAdaptationNaturalArtificial1975} proposed
the Schema Theorem arguing why the genetic approach described above works.
This subsection describes the main idea behind the argument.
First, an important term to describe is schema.\\

\navesti{Schema} is an extended representation of chromosome,
where each gene can contain a “don’t care” symbol marked as underscore $\_$.
This symbol can take up any value that an allele can in the given context.
We can then say that a chromosome belongs to a schema
and that a schema contains a chromosome.\\

Schema can be illustrated on a chromosome with one gene represented as a vector that contains a permutation of numbers $1$ to $7$.
Then, example of a schema is $H_1 = \langle 5, \_, \_, 2, \_, 3, \_ \rangle$.
It contains $24$ chromosomes, with one example being $\langle 5, 4, 1, 2, 6, 3, 7 \rangle$.
On the other hand, schema $H_2 = \langle 1, 2, 3, 4, 5, 6, \_ \rangle$ contains only one chromosome.

\newpage
There are two other properties that a schema has.
They are length and order and are defined as follows.\\

\definice{Length} of a schema is the distance from the first “non-don’t care” symbol to the last.\\

\definice{Order} of a schema is the number of “non-don’t care” symbols contained in the schema.\\

For example, $H_1$ has length $6$ and order $3$.
It is graphically illustrated in figure~\ref{fig:schema}.
On the other hand, schema $H_2$ has the same length, $6$, but higher order, which is also $6$.

With schema being defined, we can interpret any population of individuals as a pool of schemata.
It can then be reformulated that a genetic approach, which has a reproductive plan and
genetic operators, (a) creates new schemata by recombination of the one already present in the population,
(b) creates schemata that are absent in the population, and (c) keeps a history of the best schemata found.
The Schema Theorem can then be written as

\begin{equation}
    \mathrm{E}[M(H, t+1)] \geq M(H, t) \dfrac{\mu(H)}{\overline{\mu}}\left[ 1 - p_c \dfrac{\delta(H)}{l-1} - \sigma(H)
    p_m \right]\,,
    \label{eq:schema-theorem}
\end{equation}

where $M(H, t)$ is expected number of individuals whose chromosome belongs to schema $H$ in population $t$,
$\mu(H)$ is average fitness of individuals whose chromosome belongs to $H$,
$\overline{\mu}$ is average population fitness,
$\delta(H)$ is length of schema $H$ with it’s maximum length $l$,
$\sigma(H)$ is order of $H$,
$p_c$ is crossover probability, and
$p_m \ll 1$ is mutation probability.

Inequality~\ref{eq:schema-theorem} says, that the success of a schema $H$,
considering only crossover and low probability mutation are purely determined
by its better-than-average performance, length, and order.
It can be thus said that the genetic approach favors
short schemata with low order that have better-than-average performance.

The reasoning behind the argument is that schemata with high order are more likely
to be damaged by mutation, i.e., an allele of a schema is replaced by a different one.
Also, longer schemata are more likely to be split using a crossover, whereas Holland
considers a one-point-crossover that produces an offspring’s chromosome by copying
of the first parent’s chromosome up to the crossover point, followed by the second parent chromosome after the crossover point.

\begin{figure}[h]
    \includegraphics[width=0.7\textwidth, center]{schema}
    \caption[Example of a schema]{Example of a schema, where “don’t care” symbol marked as underscore $\_$.}
    \label{fig:schema}
\end{figure}
