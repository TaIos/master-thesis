\section{Dataset}\label{sec:dataset}

This section describes the parameters used to create datasets for each testing scenario in subsection~\ref{subsec:generation-parameters}.
Generated instances are described in subsection~\ref{subsec:instances}.

Additionally, as mentioned in chapter~\ref{ch:literature-review}, no datasets in the
literature would satisfy the definition~\ref{eq:painting-placement-instance} of painting placement instance.
Thus, all datasets are exclusively created by the author and can be used by other researchers
for benchmarking their solutions.

Generation of testing instances is performed using a Python programming language in combination with Jupyter Notebook.
Both the datasets and Notebooks are in the attached medium.

\subsection{Generation parameters}\label{subsec:generation-parameters}

Seven parameters are used for dataset generation.
They are layout area ratio, max painting ratio, eval function, max painting width, max painting height, flow min, and flow max.
Their values for generating instances for testing scenarios are in table~\ref{tab:generation-parameters}.
A description of each of them follows in the rest of this subsection.

\begin{table}[h!]
    \caption{Parameters used to generate testing scenarios}
    \begin{threeparttable}
        \begin{tabular}{lccccccc}
            \hline
            scenario &
            \begin{tabular}[c]{@{}c@{}}layout\\ area ratio\end{tabular} &
            \begin{tabular}[c]{@{}c@{}}max paint.\\ width\end{tabular} &
            \begin{tabular}[c]{@{}c@{}}max paint.\\ height\end{tabular} &
            \begin{tabular}[c]{@{}c@{}}max paint.\\ ratio\end{tabular} &
            \begin{tabular}[c]{@{}c@{}}flow\\ min\end{tabular} &
            \begin{tabular}[c]{@{}c@{}}flow\\ max\end{tabular} &
            \begin{tabular}[c]{@{}c@{}}eval\\ function\end{tabular} \\ \hline
            random                                                      & 1.2 & 10 & 10 & 3 & 0 & 4 & $f(x,y) = x+y$ \\ \hline
            cluster                                                     & 1.2 & 10 & 10 & 3 & - & - & $f(x,y) = 0$   \\ \hline
            packing                                                     & 1   & 10 & 10 & 3 & 0 & 4 & $f(x,y) = 0$   \\ \hline
            \begin{tabular}[c]{@{}l@{}}biased\\ clustering\end{tabular} & 1.3 & 10 & 10 & 3 & - & - & -              \\ \hline
        \end{tabular}
        \begin{tablenotes}
            \small
            \item Left-out values marked with - are discussed later in the text.
        \end{tablenotes}
    \end{threeparttable}
    \label{tab:generation-parameters}
\end{table}

\definice{Layout area ratio} is the ratio between the area of the layout and the painting area sum.
It is computed as

\[
    \dfrac{\sum\limits_{i=1}^{N} w_i h_i}{WH}\,,
\]

where $w_i$ is width, $h_i$ is height of painting $i$, $W$ is width, and $H$ is height of the layout.
If the layout area ratio is set to $1$, it means a preference for more compact layouts.
On the other hand,
increasing this value implies the presence of more free space in the resulting layout.

\definice{Max painting ratio} controls the maximum aspect ratio between width $w$ and height $h$ of each painting.
It is computed as

\[
    \dfrac{\max(w,h)}{\min(w,h)}\,.
\]

Increasing the max painting ratio implies the possibility of the generation of paintings
that are very thin, i.e., $w \ll h$ or $h \ll w$.
On the other hand, setting the value to 1
implies that every generated painting is square.

\definice{Eval function} is function $\pi$ from eq.~\ref{eq:objective}.
In the random scenario, the eval function is set to $f(x,y) = x+y$ because of its simplicity, linearity, and interpretability.
Also, it implies that it is advantageous to place small paintings close to the bottom left corner as the function value is the lowest there and
big paintings to the top right corner.
On the other hand, for clustering and packing scenarios, the eval function is set to a constant value $f(x,y) = 0$.
The reason is that different capabilities are tested (clustering and packing).
Furthermore, using a non-constant eval function makes it harder to interpret the results.

Rest of the parameters, \definice{max painting width}, \definice{max painting height}, \definice{flow min}, \definice{flow max}
are self-explanatory and were set as low numeric values to increase computation speed and avoid overflow.

\subsection{Instances}\label{subsec:instances}

Seven painting placement instances are generated using table~\ref{tab:generation-parameters} parameters.
Generated instances are described in table~\ref{tab:instances}.
For random and packing scenarios, values for max painting width/height, flow min/max, and max painting ratio are randomly generated from the parameter range described in table~\ref{tab:generation-parameters}.
The clustering and biased clustering parameters are also generated randomly.
The only exception is the flow, which is set in a way to form clusters.


Visualization of the flow for two painting placement instances is in figure~\ref{fig:instance-flow}.
On the left is the randomly generated flow for random\_10 instance, and on the right is the flow for cluster\_3\_6 instance that forms clusters.

Lastly, the eval function for biased\_sparse\_cluster\_3\_5 partitions the layout into six same-size chessboard squares.
Half are assigned the same advantageous evaluation to force the cluster formation.
The second half is evaluated to zero.
The concrete value for the eval function is in the appendix in listing~\ref{lst:biased-sparse-cluster-eval-function}.

\begin{table}[h!]
    \caption{Values of generated instances}
    \label{tab:instances}
    \begin{tabular}{lcccc}
        \hline
        instance name &
        \begin{tabular}[c]{@{}c@{}}paint.\\ count\end{tabular} &
        \begin{tabular}[c]{@{}c@{}}layout\\ width, height\end{tabular} &
        scenario &
        description \\ \hline
        random\_10    & 10 & 24 x 19 & random     &                                                                        \\ \hline
        random\_20    & 20 & 31 x 25 & random     &                                                                        \\ \hline
        packing\_10   & 10 & 19 x 15 & packing    &                                                                        \\ \hline
        packing\_20   & 20 & 33 x 26 & packing    &                                                                        \\ \hline
        cluster\_3\_6 & 18 & 30 x 25 & clustering & \begin{tabular}[c]{@{}c@{}}3 clusters,\\ 6 paintings each\end{tabular} \\ \hline
        cluster\_4\_5 & 20 & 34 x 27 & clustering & \begin{tabular}[c]{@{}c@{}}4 clusters,\\ 5 paintings each\end{tabular} \\ \hline
        biased\_sparse\_cluster\_3\_5 &
        15 &
        29 x 23 &
        biased clustering &
        \begin{tabular}[c]{@{}c@{}}3 clusters,\\ 5 paintings each\end{tabular} \\ \hline
    \end{tabular}
\end{table}



