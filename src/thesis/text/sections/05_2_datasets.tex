\section{Dataset}\label{sec:dataset}

This section describes the parameters for creating datasets for random, clustering, and packing testing scenarios.
Generated painting placement instances are described in subsection~\ref{subsec:instances}.

Additionally, as mentioned in chapter~\ref{ch:literature-review}, no datasets in the
literature would satisfy the definition~\ref{eq:painting-placement-instance} of painting placement instance.
Thus, all datasets are exclusively created by the author and can be used by other researchers
for benchmarking their solutions.

Generation of testing instances is performed using a Python programming language in combination with Jupyter Notebook.
Both the datasets and Notebooks are in the attached medium.

\subsection{Generation parameters}\label{subsec:generation-parameters}

Generation parameters used to create testing instances are presented in table~\ref{tab:generation-parameters}.
A description of each of them follows in the rest of this subsection.

\begin{table}[h!]
    \caption{Parameters used to generate testing scenarios}
    \begin{threeparttable}
        \begin{tabular}{llllllll}
            \hline
            \textbf{} &
            \textbf{\begin{tabular}[c]{@{}l@{}}Layout\\ area ratio\end{tabular}} &
            \textbf{\begin{tabular}[c]{@{}l@{}}Max paint.\\ width\end{tabular}} &
            \textbf{\begin{tabular}[c]{@{}l@{}}Max paint.\\ height\end{tabular}} &
            \textbf{\begin{tabular}[c]{@{}l@{}}Max paint.\\ ratio\end{tabular}} &
            \textbf{\begin{tabular}[c]{@{}l@{}}Flow\\ min\end{tabular}} &
            \textbf{\begin{tabular}[c]{@{}l@{}}Flow\\ max\end{tabular}} &
            \textbf{\begin{tabular}[c]{@{}l@{}}Eval\\ func.\end{tabular}} \\ \hline
            random     & 1.2 & 10 & 10 & 3 & 0 & 4 & $x+y$ \\ \hline
            clustering & 1.2 & 10 & 10 & 3 & - & - & $0$   \\ \hline
            packing    & 1   & 10 & 10 & 3 & 0 & 4 & $0$   \\ \hline
        \end{tabular}
        \begin{tablenotes}
            \small
            \item Left-out values marked with - are discussed later in the text.
        \end{tablenotes}
    \end{threeparttable}
    \label{tab:generation-parameters}
\end{table}

\definice{Layout area ratio} is the ratio between the area of the layout and the painting area sum.
It is computed as

\[
    \dfrac{\sum\limits_{i=1}^{N} w_i h_i}{WH}\,,
\]

where $w_i$ is width, $h_i$ is height of painting $i$, $W$ is width, and $H$ is height of the layout.
If the layout area ratio is set to $1$, it means a preference for more compact layouts.
On the other hand,
increasing this value implies the presence of more free space in the resulting layout.

\definice{Max painting ratio} controls the maximum aspect ratio between width $w$ and height $h$ of each painting.
It is computed as

\[
    \dfrac{\max(w,h)}{\min(w,h)}\,.
\]

Increasing the max painting ratio implies the possibility of the generation of paintings
that are very thin, i.e., $w \ll h$ or $h \ll w$.
On the other hand, setting the value to 1
implies that every generated painting is square.

\definice{Evaluation function} is function $\pi$ from objective function~\ref{eq:objective}.
In the random scenario, the evaluation function is set to $x+y$ because of its simplicity, linearity, and interpretability.
Also, it implies that placing small paintings close to the bottom left corner is advantageous as the function value is the lowest there and
big paintings to the top right corner.
On the other hand, for clustering and packing scenarios, the evaluation function is set to a constant value $0$.
The reason is that different capabilities are tested (clustering and packing).
Furthermore, using a non-constant evaluation function makes it harder to interpret the results.

Rest of the parameters, \definice{max painting width}, \definice{max painting height}, \definice{flow min}, \definice{flow max}
are self-explanatory and were set as low numeric values to increase computation speed and avoid overflow.

\subsection{Instances}\label{subsec:instances}

Six painting placement instances are generated using table~\ref{tab:generation-parameters} parameters, and one is created from the London National Gallery in figure~\ref{fig:london-wall}.
They are described in table~\ref{tab:instances}.
For random and packing scenarios, values for max painting width/height, flow min/max, and max painting ratio are randomly generated from the parameter range described in table~\ref{tab:generation-parameters}.
The clustering parameters are also generated randomly.
The only exception is the flow, which is set in a way to form clusters.

Visualization of the flow for two painting placement instances is in the appendix in figure~\ref{fig:instance-flow}.
On the left is the randomly generated flow for random\_10 instance, and on the right is the flow for cluster\_3\_6 instance that forms clusters.

\begin{table}[h!]
    \caption{Painting placement instances}
    \label{tab:instances}
    \begin{tabular}{lllll}
        \hline
        \textbf{Instance name} &
        \textbf{\begin{tabular}[c]{@{}l@{}}Paint.\\ count\end{tabular}} &
        \textbf{\begin{tabular}[c]{@{}l@{}}Layout\\ width, height\end{tabular}} &
        \textbf{Scenario} &
        \textbf{Description} \\ \hline
        random\_10    & 10 & 24 x 19 & random     &                                                                        \\ \hline
        random\_20    & 20 & 31 x 25 & random     &                                                                        \\ \hline
        packing\_10   & 10 & 19 x 15 & packing    &                                                                        \\ \hline
        packing\_20   & 20 & 33 x 26 & packing    &                                                                        \\ \hline
        cluster\_3\_6 & 18 & 30 x 25 & clustering & \begin{tabular}[c]{@{}l@{}}3 clusters,\\ 6 paintings each\end{tabular} \\ \hline
        cluster\_4\_5 & 20 & 34 x 27 & clustering & \begin{tabular}[c]{@{}l@{}}4 clusters,\\ 5 paintings each\end{tabular} \\ \hline
        london\_gallery\_wall &
        7 &
        180 x 90 &
        \begin{tabular}[c]{@{}l@{}}London\\ National\\ Gallery\end{tabular} &
        \begin{tabular}[c]{@{}l@{}}created from\\ figure~\ref{fig:london-wall}\end{tabular} \\ \hline
    \end{tabular}
\end{table}
