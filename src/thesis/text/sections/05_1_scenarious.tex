\section{Scenarios}\label{sec:scenarios}

Four testing scenarios evaluate different aspects of the proposed solution.
They are random, clustering, packing, and biased clustering.
Additionally, one scenario describes the painting placement at the London National Gallery in figure~\ref{fig:london-wall}. \\

\navesti{Random} scenario contains randomly generated painting placement instances.
It is mainly used for performance testing.
\\

\navesti{Clustering} scenario tests the ability to form clusters.
It is achieved by dividing the paintings into groups.
Paintings belonging to the same group have increased flow between them.
Paintings from the distinct group have flow between them set to 0.
\\

\navesti{Packing} scenario is the same as the random scenario, with the only difference being that the layout area is equal to the area of all paintings summed together.
It tests the ability to create compact solutions.
\\

\navesti{Biased clustering} scenario is the same clustering scenario with the only difference
that the evaluation function $\pi$ (eq.~\ref{eq:objective})
is set in a way that the formation of clusters is advantageous in certain parts of the layout.
It tests the ability to create clusters and force their formation in certain parts of the layout.\\
