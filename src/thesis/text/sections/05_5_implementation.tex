\clearpage% Flush earlier floats (otherwise order might not be correct)
\newpage


\section{Implementation}\label{sec:implementation}

The proposed implementation of a genetic approach is written in Java 11
using a Play Framework v2.8\footnotemark[1], a web framework for Java and Scala.

Implementation behaves like a computation server to which
a user can submit a computation.
Then, the server asynchronously starts the submitted computation and returns an identifier of the computation.
It means that multiple computations can be submitted without blocking the user.
The user can then check the computation state using the returned identifier.

To start the computation server, locate the directory containing a file \verb|build.sbt| in the attached medium (see appendix~\ref{ch:contents-of-the-attached-medium}).
Then, run the following command in that directory (Java~11, SBT\footnotemark[3], and Scala must be installed).

\begin{listing}[h!]
    \begin{cminted}[autogobble,breaklines=true]{shell}
        $ sbt run
    \end{cminted}
    \caption[Starting a computation server]
    {Starting a computation server.}
    \label{lst:sbt-run}
\end{listing}

Command in listing~\ref{lst:sbt-run} uses \verb|sbt|\footnotemark[3] with \verb|run| argument to start the computation server.
By default, the computation server accept requests on \verb|localhost:9000|.
An example of submitting a computation with a predefined instance name to the computation server is in listing~\ref{lst:computation-submission-dataset}.
An example of a successful computation submission response is in listing~\ref{lst:computation-response-success}.

\begin{listing}[h!]
%    \centering
    \begin{cminted}[autogobble,breaklines=true]{json}
    {
        "id":"random_10_9B5F8",
        "outputDirectory":"./out/088_random_10_9B5F8"
    }
    \end{cminted}
    \caption[Successful computation submission response]
    {Successful computation submission response.}
    \label{lst:computation-response-success}
\end{listing}

The computation server also validates input before starting the computation.
For example, if misspelling the instance name, the response by the computation server can be seen in~\ref{lst:computation-response-fail}.

\begin{listing}[h!]
    \centering
    \begin{cminted}[autogobble,breaklines=true]{json}
    {
        "message":"Entity [DatasetDto] with identifier [randomm_10] was not found."
    }
    \end{cminted}
    \caption[Unsuccessful computation submission response]
    {Unsuccessful computation submission response.}
    \label{lst:computation-response-fail}
\end{listing}

Lastly, there is also an option not to specify the instance name in the submission, as seen
in listing~\ref{lst:computation-submission-dataset} where the instance name is random\_10.
In that case, a user has to specify the instance manually in the request – layout width and height,
paintings together with their flow and evaluation function $\pi$ (eq.~\ref{eq:objective}), in the format that is accepted by
mXparser\footnotemark[4].
An example of submission without specifying the instance name is in the appendix in listing~\ref{lst:computation-submission-manual}.

% @formatter:off
\begin{listing}[h!]
\centering
\begin{minted}[autogobble,breaklines=true]{shell}
$ curl --location 'localhost:9000/compute/dataset' \
--header 'Content-Type: application/json' \
--data '{
  "datasetName": "random_10",
  "gaParameters": {
    "maxNumberOfIter": 300,
    "populationSize": 500,
    "maximumWildCardCount": 1,
    "orientationWeights": [
      1,
      1,
      0.5
    ],
    "geneticAlgorithm": "simpleGa",
    "mate": "normalizedProbabilityVectorSum",
    "mutate": "flipOnePartAtRandom",
    "select": "tournament",
    "objective": "simple",
    "evaluator": "ga",
    "placingHeuristics": "corner",
    "populationDivisionCounts": {
      "elite": 0.2,
      "average": 0.6,
      "worst": 0.2,
      "children": 0.3,
      "mutant": 0.2,
      "winner": 0.2,
      "random": 0.1
    },
    "initialPopulationDivisionCounts": {
      "random": 0.7,
      "greedy": 0.3
    }
  },
  "objectiveParameters": {
    "name": "simple",
    "params": {
      "overlappingPenalizationConstant": 30.61,
      "outsideOfAllocatedAreaPenalizationConstant": 30.61
    }
  }
}'
\end{minted}
\cprotect\caption[Example of computation submission with instance name]
{Example of computation submission of random\_10 instance using \verb|curl|\footnotemark[2] to a computation server running on \verb|localhost:9000|.}
\label{lst:computation-submission-dataset}
\end{listing}
% @formatter:on

\footnotetext[1]{\url{https://www.playframework.com/documentation/2.8.x/Home}}
\footnotetext[2]{\url{https://curl.se/}}
\footnotetext[3]{\url{https://www.scala-sbt.org/}}
\footnotetext[4]{\url{https://mathparser.org/}}
