\section{Scenarios}\label{sec:scenarios}

There are 4 scenarios that are tested – random, clustering, packing and biased clustering.
Also, there is one dataset that describes the painting placement at the London National Gallery in figure~\ref{fig:london-wall}. \\

\navesti{Random} scenario contains randomly generated datasets that are mainly used for performance testing.\\

\navesti{Clustering} scenario tests the ability to form clusters.
This is achieved as dividing the paintings to groups. Paintings belonging to the
same group have increased flow between them. Paintings from the distinct group have flow between
them set to 0. \\

\navesti{Packing} scenario is the same as the \textit{random} with the only difference that the layout area
is equal to the area of all paintings summed together. This tests the ability to create compact solutions.\\

\navesti{Biased clustering} scenario is similar to the \textit{clustering} with the only difference
that the evaluation function is set in a way that the formation of clusters is advantageous in certain parts
of the layout. This test the ability to not only create clusters but also to force their formation in certain parts of the layout.\\