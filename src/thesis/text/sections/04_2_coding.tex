\section{Coding}\label{sec:coding}

The central part of the novel genetic approach proposed in this thesis is how an individual is represented.
It is crucial for constructing the genetic operators, e.g., crossover and mutation, and decoding an individual from its representation to the solution.

An individual is represented as a 3D chromosome—which means having three genes—that is composed of
painting sequence random key, slicing order random key, and orientation probabilities.
An example of a chromosome is in figure~\ref{fig:chromosome}.

Let us use the notation for painting sequence random key as $PS_{rk}$,
slicing order random key as $SO_{rk}$,
orientation probabilities as $OR_{prob}$ and instance size as $N$, i.e., number of paintings.
First two are vectors, where $PS_{rk} \in \real^N$ and $SO_{rk} \in \real^{N-1}$.
Orientation probabilities is a matrix where $OR_{prob} \in \real^{N-1, 3}$.
Constraints~\ref{eq:constraints} apply to each of these parts with
a \definice{stochastic vector} defined as a vector that contains non-negative elements that add up to one.

\begin{equation}
    \begin{aligned}
        & 1. \quad PS_{rk} \text{ is a stochastic vector.} \\
        & 2. \quad SO_{rk} \text{ is a stochastic vector.} \\
        & 3. \quad \text{Each row in } OR_{prob} \text{ is a stochastic vector.}
    \end{aligned}\label{eq:constraints}
\end{equation}


The representation mentioned above is based on a genetic solution to the FLP
from~\cite{friedrichIntegratedSlicingTree2018, riponAdaptiveVariableNeighborhood2013},
where the authors represent an individual as a 3D chromosome with concrete identifiers for facilities, slicing order, and orientations.
The novel approach to coding proposed in this thesis is
(1) the use of stochastic vectors instead of concrete identifiers,
(2) interpreting the stochastic vectors as random keys~\cite{beanGeneticAlgorithmsRandom1994},
(3) novel mutation and crossover operator, and
(4) decoding an individual from the stochastic vector representation.
Thus, the search in the proposed genetic approach takes place in a different space, which is a space of stochastic vectors.

\begin{figure}[htp]
    \includegraphics[width=0.8\textwidth, center]{chromosome}
    \caption[Example of an individual representation]{
        Example of an individual representation – two vectors and one matrix.
        Each vector and matrix row form a stochastic vector (vector that contains non-negative elements that add up to one).
    }
    \label{fig:chromosome}
\end{figure}

There are multiple ideas behind representing an individual as a set of stochastic vectors that stem
from extending RKGA~\cite{beanGeneticAlgorithmsRandom1994}, in which chromosome is represented as a vector with elements from interval $\langle0,1\rangle$.
First of them is the ability to perform mutation at an arbitrary element of these vectors using
a simple replacement, i.e., substituting an element for a random one from interval $\langle0,1 \rangle$,
followed by normalization back to the stochastic vector.
For example, using representation described in~\cite{friedrichIntegratedSlicingTree2018, riponAdaptiveVariableNeighborhood2013},
there must be a different mutation method for each gene of a chromosome.
By using the representation proposed in this thesis, there has to be only one mutation operator that can be used universally
for all genes of the chromosome.

Additionally, when using representations similar to~\cite{friedrichIntegratedSlicingTree2018, riponAdaptiveVariableNeighborhood2013}, after applying the genetic operators, usually crossover and mutation, an invalid individual might be created.
That is an individual that does not represent any solution.
The presence of invalid individuals might lead to performance loss in FLP~\cite{liuMultiimprovedGeneticAlgorithm2012}.
Moreover, unique solutions for dealing with invalid individuals must be introduced.
For example, left-to-right scan used by~\cite{hwangGeneticAlgorithmApproach2009, kandasamyEffectiveLocationMicro2020} or leaving invalid individuals inside the population but penalizing them~\cite{hwangGeneticAlgorithmApproach2009}.
The coding proposed in this thesis produces only valid individuals.

Finally, the reasoning behind using a stochastic vector instead of a vector, where each element is from interval $\langle0,1\rangle$ as in RKGA~\cite{beanGeneticAlgorithmsRandom1994}, is the novel implementation of crossover used in this thesis.
It is described in subsection~\ref{subsec:crossover}.
