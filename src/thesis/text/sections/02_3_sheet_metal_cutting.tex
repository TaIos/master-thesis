\section{Sheet metal cutting}\label{sec:sheet-metal-cutting}

The sheet metal cutting problem evaluates two aspects – compactness of the layout, also called pattern efficiency, and distance taken
by the path-cutting tool to cut all placed parts.
Also, placed parts that determine the pattern efficiency can be irregular, for example, forming a polygon~\cite{hajadLaserCuttingPath2019}.

Authors in~\cite{vijayanandHeuristicGeneticApproach2015}
use both of these aspects to formulate an objective function $F$ that defines a sheet metal cutting problem as

\begin{equation}
    \min F = w_1\,f_1 + w_2\,f_{2}\,,
    \label{eq:metal-sheet-cutting}
\end{equation}

where $f_1$ is pattern efficiency, which is calculated as $\dfrac{\sum\limits_{i=1}^{N}a_i}{WH}$,
with $N$ being number of placed parts, $a_i$ the area of the $i$-th placed part, $W$ width and $H$ height of the
the smallest rectangle that can contain all placed parts, and $f_2$ the distance
needed by the path-cutting tool to cut all the placed parts.
Weights $w_1 \in \realpos, w_2 \in \realpos$ balance the bias towards pattern efficiency or path-cutting tool distance.

Some variants of the sheet metal cutting problem neglect the pattern efficiency and try to only find the shortest
path of a given placed parts~\cite{kandasamyEffectiveLocationMicro2020}.
Then, the sheet metal cutting problem can be reformulated as the Generalized Traveling Salesman Problem (GTSP)~\cite{hajadLaserCuttingPath2019}.
The sheet metal cutting problem is thus considered to be NP-hard problem~\cite{vijayanandHeuristicGeneticApproach2015}.

Authors in~\cite{vijayanandHeuristicGeneticApproach2015} solve sheet metal cutting problem using a genetic algorithm
where an individual is represented as a 3D chromosome.
It contains the cluster size written as a binary number, the sequence of placed parts, and their orientation.
This 3D chromosome is then input to the placing heuristic, which places the parts.

The solution proposed in~\cite{kandasamyEffectiveLocationMicro2020} considers only path-cutting tool distance.
However, they partition each placed part into multiple segments using micro joints, i.e., points at the edges of the part.
The path-cutting tool then proceeds by cutting these segments instead of the whole part at once.
They propose a genetic algorithm with the 2D chromosome – containing angles that determine the placement of the micro joints and cutting sequence of these segments.
The chromosome is then directly used by the path-cutting tool that starts cutting segments according to the cutting sequence defined in the second part of the chromosome.