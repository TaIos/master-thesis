\newpage


\section{Implementation}\label{sec:implementation}

Proposed implementation of a genetic approach is written in Java 11
using a Play Framework\footnotemark[1] v2.8, which is a web framework for Java and Scala.

Implementation using a Play Framework behaves like a computation server to which
a user can submit a computation.
Then, the server asynchronously starts the computation and returns an identifier of the computation.
User can then check the computation state using this identifier.

Steps to submit a computation are as follows.

\begin{enumerate}
    \item a
    \item b
    \item c
\end{enumerate}




\begin{listing}[h!]
    \centering
    \begin{cminted}[autogobble,breaklines=true]{shell}
        $ curl --location 'localhost:9000/compute/dataset' \
        --header 'Content-Type: application/json' \
        --data '{
            "datasetName": "random_10",
            "gaParameters": {
                "maxNumberOfIter": 300,
                "populationSize": 100,
                "maximumWildCardCount": 1,
                "orientationWeights": [ 1, 1, 0.5 ],
                "geneticAlgorithm": "simpleGa",
                "mate": "normalizedProbabilityVectorSum",
                "mutate": "flipOnePartAtRandom",
                "select": "tournament",
                "objective": "simple",
                "evaluator": "ga",
                "placingHeuristics": "corner",
                "populationDivisionCounts": {
                    "elite": 0.2,
                    "average": 0.6,
                    "worst": 0.2,
                    "children": 0.3,
                    "mutant": 0.2,
                    "winner": 0.2,
                    "random": 0.1
                },
                "initialPopulationDivisionCounts": {
                    "random": 0.7, "greedy": 0.3
                }
            },
            "objectiveParameters": {
                "name": "simple",
                "params": {
                    "overlappingPenalizationConstant": 1000,
                    "outsideOfAllocatedAreaPenalizationConstant": 0
                }
            }
        }'
    \end{cminted}
    \cprotect\caption[Example of computation submission]{Example of computation submission of random\_10 instance using \verb|curl|\footnotemark[2] to a server running on \verb|localhost:9000|.}
    \label{lst:computation-submission}
\end{listing}

\footnotetext[1]{\url{https://www.playframework.com/documentation/2.8.x/Home}}
\footnotetext[2]{\url{https://curl.se/}}
